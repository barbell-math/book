\section{Bias Function}
The bias function needs to capture the users current abilities from all the data given to the model. As with the previous two sections, a bias function is needed for a \textit{single lift}. The model can then be generally applied to as many lifts as needed.

When thinking about the bias function it has to be considered within the context of the error equation in the previous section. In order to capture the users current abilities the bias function will need to be $< 1$ for values that are considered current, reducing the errors from those data points. For values that are considered as not current the bias function will need to be $\ge 1$, increasing the errors from those data points. With this modification of errors the potential surface will be fit current data points better than non-current data points.

The phrase "current abilities" is rather ambiguous. Obviously, there is no one point that can define what is current and what is not for a user. The closest to discrete points in time being able to define what is current is the length of rotations. As such, current will be defined as $r_{ot}$, the length of the current rotation, weeks from the current date.

At the start of a new rotation data points from the users previous rotation will be used, assuming they are within $r_{ot}$ weeks of the current date. This makes sense because that data is the closest to representing the users current abilities. As the user continues their current rotation, data from the new rotation will supplant data from the old rotation, allowing for changes in the users current abilities to be reflected in the model. If, at the start of a rotation, the data from the user does not fall within $r_{ot}$ weeks from the current date then that data will be assigned a high error bias, which makes sense because abilities fade with time and a lack of practice.

A relationship between the time since a lift was performed and the bias of that lift is needed, $w(t)$.

The model will be given:
\begin{itemize}
    \item Historical lifting data: $\{(r,s,t,p)_i\}_{i=0}^{n}$ where:
    \begin{itemize}
        \item $r\equiv$ reps
        \item $s\equiv$ sets
        \item $p\equiv$ percentage of the appropriate 1RM that the sets and reps were performed at
        \item $t\equiv$ how long ago the lift was performed
    \end{itemize}
\end{itemize}

The bias function is shown below. An inverse cubic function is centered at $(r_{ot},1)$ and scaled appropriately to ensure that no biases are overly large or small.

\begin{equation}
    w\left(t\right)=r_{ot}^{-1}\left(t-r_{ot}\right)^{\frac{1}{3}}+1
\end{equation}

An inverse cubic function was chosen because is is not limited asymptotically, which means that the bias weights will not be limited to some constant as the data points get older. It was also chosen because the rate of increase is easily controlled to prevent biases that are overly large or small within any reasonable time frame.

\begin{equation*}
    \lim_{t \to +\infty}w(t)=
    \lim_{t \to +\infty}r_{ot}^{-1}\left(t-r_{ot}\right)^{\frac{1}{3}}+1=\infty
\end{equation*}