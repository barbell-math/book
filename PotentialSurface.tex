\part{
    Potential Surface
    \\
    \large{Establishing What's Possible}
}
\label{sec:PotentialSurface}

\chapter{Initial Explorations}

As stated in the introduction, every person responds to lifting differently. This chapter is concerned with finding and establishing the limits of a given lifter within the context of a single exercise.

There are many limiting factors that need to be considered. How many reps can be done per set? How many sets can be done per exercise? What weight can the user lift given a particular amount of sets and reps? How much effort should the lifter exert to perform those sets and reps at the given weight? The answers to these questions form the thought process for the rest of the section.

When performing a single exercise, there are generally six things that dictate what is done:
\begin{enumerate}
    \item What exercise is being performed
    \item The number of sets being performed for a particular exercise
    \item The number of reps being performed for each set
    \item The weight each rep should be performed at
    \item The amount of effort expected to be exerted on each set
    \item The lifters current abilities
\end{enumerate}

The exercise being performed is largely dependent on what goals the lifter is pursuing as well as what weaknesses the lifter has. Due to the complexity of exercise selection as well as it's independence from the last five items on the list, it is explored in greater detail in a separate chapter. The last five items on that list have a large amount of dependence upon each other, and the relationship among them will be established in this chapter.

\section{Intuitive Relationships Between Variables}
\label{sec:PotentialSurfaceIntuitiveRelationshipsBetweenVariables}

The ultimate goal of this part of this part is to create a surface that represents what is possible for a lifter to do. To accomplish this, the nature of the surface must first be explored. This can be done by looking at the data as well as applying already established facts about lifting.

Before moving forward it is worth mentioning why this process is being done manually instead of through other processes common in machine learning. There are two reasons why traditional machine learning methods are not used. The first reason relates to the opaque nature of machine learning. When processes similar to machine learning are conducted the nature of the surface is unknown, which can lead to problems when trying to verify the surface has the desired behavior. This is a known issue with machine learning, with it often being compared to a black box. Understanding the behavior of the surface is important to be able to constrain it so that it will not instruct a lifter to perform impossible tasks that will likely lead to injury. Being able to prove that the surface will not stray into the impossible is very important, and will require understanding of all the variables that define its behavior. The second reason is due to the sparse nature of the data. This surface will be fitted to a single exercise performed by a single lifter. This is done so the model can adapt to every lifter individually but an unfortunate consequence of this is very sparse data. Even if a lifter performs a lift three times a week with each day having a top set followed by back down sets, this will only create six data points a week. This translates to $312$ data points over an entire year, which is not much especially when the number of dimensions increases.\footnote{The curse of dimensionality strikes again...} Running traditional machine learning on a data set this small would result in little more than noise. One way to fix this would be to generalize the surface to use data from many lifters. However, this would force the surface to no longer be tailored to an individual lifter, which would defeat the entire purpose of this book. Instead, the surface will be defined in a way that restricts it's behavior to what is reasonable from the start. This will allow a small data set to be used while still tailoring the surface to an individual lifter. Without further ado, the process of defining the surface can begin.

Many of the variables that need to be looked at are extremely interdependent on each other. Sets and reps depend on each other, with an increase in one generally requiring a decrease in the other. Intensity depends on sets and reps, with higher intensities requiring possibly less sets and possibly reps. Going back to the relation to sets and reps however, if sets increases and weight decreases then reps could increase, decrease, or remain the same. Effort depends on intensity as well as sets and reps... The point should be clear. Trying to define behaviors based on all of these interdependent variables quickly gets out of hand. Instead of doing this, the behaviors of each variable will be defined in relation to volume where each variable can be viewed independently from the others. Doing this also allows more abstract properties to be considered, such as volume tolerance. With that said, the rest of this section outlines the desired behaviors of the surface.

\subsection{Property 1: Volume and Intensity}
\label{sec:PotentialSurfaceIntuitiveRelationshipsBetweenVariablesVolumeAndIntensity}

A lifter has a \textit{volume tolerance}, which represents there work capacity. Given a certain volume tolerance, the only five variables that can change are weight, intensity, effort, sets, and reps. This is demonstrated in equations \ref{eq:BaseVolumeEquation} and \ref{eq:IntensityBasedVolumeEquation}.

Weight and intensity are synonymous through the lens of volume, as the total amount of weight lifted is calculated regardless of the differing parameters. Looking at equations \ref{eq:BaseVolumeEquation} and \ref{eq:IntensityBasedVolumeEquation} it is natural to assume volume will increase with increased weight and intensity. Mathematically, this makes sense, but the limits of the human body need to be considered. As intensity approaches $100\%$ more effort is required to maintain the same amount of volume. Because effort is limited, volume will necessarily decrease at higher weights, to a point where it reverses the increase in volume from the increase in weight. An example may make this clearer. At $100\%$ intensity a lifter can realistically expect to obtain a maximum volume near there 1RM, as shown in equation \ref{eq:VolumeWithDiffereingIntensities1}. \footnote{The actual volume reached as intensity increases may be slightly more or less than the lifters 1RM because of other limiting factors such as effort and fatigue, which will be discussed in the following sections. As the next sections will explore, if effort is maximized and fatigue is minimized then the maximum possible volume should be expected to be at or above the lifters 1RM.} At $50\%$ intensity, all it takes is two reps to match the volume performed at $100\%$ intensity, which is shown in equation \ref{eq:VolumeWithDiffereingIntensities2}. At $50\%$ intensity, far more than one set of two reps can be performed, making it easy to surpass the volume done at $100\%$ intensity. The data discussed in chapter \ref{sec:DataSection} has many examples of this.

\begin{subequations}
    \begin{align}
        \label{eq:VolumeWithDiffereingIntensities1}
        v_{I=1}=&v(s=1,r=1,w=l_{1RM})=l_{1RM} \\
        \label{eq:VolumeWithDiffereingIntensities2}
        v_{I=0.5}=&v(s=1,r=2,w=0.5l_{1RM})=l_{1RM}
    \end{align}
\end{subequations}

The fact that volume diminishes at higher intensities cannot be stressed enough, as it breaks any intuition gathered from equations \ref{eq:BaseVolumeEquation} and \ref{eq:IntensityBasedVolumeEquation}. These equations, serve as a way to \textit{measure} volume, not as a proxy to determine what is possible. These equations have no regard to the limitations of the human body. It is common knowledge among strength coaches that if you want to increase volume sets and reps need to be increased, not weight.

Now, for lower intensities, it is tempting to say that as weight approaches $0$ volume will linearly decrease to $0$. Again, while mathematically true, the human body does not follow a linear decrease in volume. When weight diminishes it become easier to lift the weight. Common knowledge dictates that performing some small, light task it can be done many times. Now, this does not come at no cost, as when intensity decreases more endurance is required. However, generally speaking, the human body is made for endurance. Running for long distances is one of our strengths thanks to evolution. What this means for lifting is that as weight decreases so many more sets and reps can be done that the decrease in volume from the decrease in weight is overcome. Volume cannot indefinitely increase, there is a limit of course. This limit is known as the lifters volume tolerance. As weight decreases more sets and reps can be done until this limit is reached.

Given the above discussion the following properties can be gathered:

\begin{itemize}
	\item Property 1.a:  As intensity approaches $100\%$  volume will decrease until it is near the lifters 1RM.
	\item Property 1.b: As intensity approaches $0\%$ volume will approach some plateau, or asymptote in mathematical terms. The exact value of the plateau will need to be found from the data.
\end{itemize}

To confirm this behavior the data will need to be considered. Looking at figure \ref{fig:IntensityVsVolumeGraph}, it does not demonstrate this pattern very well. The first item of the above list is visible in the data as volume decreases past $80$\% intensity until it reaches the volume equivalent to a 1RM. The second item of the list however does not appear to be present in the data. Instead of reaching a rough plateau it reaches a peak and drops off linearly past the peak. This seemingly goes against the pattern that was discussed when considering the limits of the human body.

\begin{figure}
    \centering
    \includegraphics[scale=0.55]{images/ch3/IntensityVsVolume.png}
    \caption{A graph comparing intensity and volume. Note how there is a clear peak in volume between $60\%$-$80\%$.} 
    \label{fig:IntensityVsVolumeGraph}
\end{figure}

The explanation for the peak in volume is rather simple. It is due to the data being collected by a powerlifter, a sport that does not require much in the way of endurance, which skews the data as a result. This skewing of the data will come up again and be discussed more thoroughly in the next section, which is concerned with volume and effort.


\subsection{Property 2: Volume and Effort}
\label{sec:PotentialSurfaceIntuitiveRelationshipsBetweenVariablesVolumeAndEffort}

The next question is what happens when effort increases or decreases across all intensities. As effort increases more sets will be able to be completed, more reps will be able to be completed in each set, or more weight will be able to be lifted. If the increase in effort is large enough, some combination of sets, reps, and weight could all increase. The opposite is true if effort decreases. Again, it should be clear that sets and reps are directly proportional to volume, implying an increase in either one will increase volume. Putting all of this together, as effort increases volume will increase and as effort decreases volume will decrease.

\begin{figure}
    \centering
    \includegraphics[scale=0.55]{images/ch3/EffortVsVolume.png}
    \caption{A graph comparing effort and volume. Note how there appears to be no correlation between volume and effort.}
    \label{fig:EffortVsVolumeGraph}
\end{figure}
\begin{figure}
    \centering
    \includegraphics[scale=0.55]{images/ch3/IntensityVsEffort.png}
    \caption{A graph comparing effort and intensity. Note how the intensity tends to increase with higher effort values. This is the result of a powerlifters goal to increase a 1RM.}
    \label{fig:EffortVsIntensity}
\end{figure}

However, in figure \ref{fig:EffortVsVolumeGraph}, which compares volume and effort, volume does not seem to correlate with effort as previously mentioned. Only at the extreme values of each RPE rating does the behavior seem to follow. When looking at the graph as a whole there appears to be no pattern. The reason for this is the same as the previous section. A powerlifter collected the data and the goal of a powerlifter, above all, is to maximize weight. This goal can be seen in figure \ref{fig:EffortVsIntensity}, where intensity positively correlates with effort. What figure \ref{fig:EffortVsIntensity} is saying is that as effort increases the lifter chose to increase weight over sets and reps. Increasing weight, or intensity, will not increase volume as much as if sets and reps were increased due to the previously established fact that volume at higher intensities is necessarily limited by effort. This is a clear bias in the data set, and explains why volume is seemingly constant in figure \ref{fig:EffortVsVolumeGraph}. Given a sport that is more concerned with rep maxes, such as crossfit, the effort vs intensity graph would level out because lower intensities would be pushed for as many reps as possible, creating maximal effort sets with lower intensities. \footnote{This is a perfect example to demonstrate how effort and intensity are two different concepts.} In order to have maximal effort sets with lower intensities either sets, reps, or both sets and reps will need to increase, forcing volume to increase considerably and restoring the correlation between effort and volume previously discussed.


\subsection{Property 3: Volume and Fatigue}
\label{sec:PotentialSurfaceIntuitiveRelationshipsBetweenVariablesVolumeAndFatigue}

Fatigue will necessarily limit volume. The reasoning behind this is simple, the more fatigued you are the less work you can safely do. There can be many sources of fatigue but they will all have the same effect of limiting volume. At the extreme, fatigue will make it so no work can be done at all and hence $0$ volume can be tolerated. This is a state that should be be avoided at all costs, as it is the epitome of over training. When fatigue is at a minimum a lifter will be able to reach there highest potential in there current state. Note that fatigue cannot add to a lifters abilities, it can only diminish them. This is true regardless of effort. No matter how much effort a lifter puts into a set, they would have always been able to lift more weight had they been less fatigued.


\subsection{Property 4: Sets and Reps}
\label{sec:PotentialSurfaceIntuitiveRelationshipsBetweenVariablesSetsAndReps}

With effort, intensity, and fatigues relation to volume being explored, sets and reps are all that's left. It may be tempting to conclude that volume should be constant given a particular weight and effort. This conclusion can be reached by looking at equations \ref{eq:BaseVolumeEquation} and \ref{eq:IntensityBasedVolumeEquation}, where once the weight and effort are known sets and reps would just vary inversely to ensure volume remains constant. Again, while mathematically true, the limits of the human body need to be considered. Evidence that the relationship between sets and reps is not perfectly inverse is shown in figure \ref{fig:SetsVsReps}, where a slight drop in sets from the expected inverse pattern is seen. This favoritism between sets and reps will of course mean volume is no longer constant given a particular intensity and effort, and as such will be known as the \textit{volume skew}.
%This can be attributed to a lack of \textit{endurance}, where sets with a greater number of reps  require more endurance than sets with less reps. Powerlifters are eternally known for having no endurance, so it should come as no surprise that they will favor more sets over reps.

\begin{figure}[h]
    \centering
    \includegraphics[scale=0.55]{images/ch3/SetsVsReps.png}
    \caption{A graph comparing average reps and sets for main compound and main compound accessory lifts. Note how the relationship is not perfectly inverse. Don't assume any volume skew seen in this graph is necessarily true for all the data, as this plot shows many exercises over a large time span. Finding volume skew will eventually need to be done on a per exercise basis across time.}
    \label{fig:SetsVsReps}
\end{figure}

\subsection{Property 5: Bounded Volume}
\label{sec:PotentialSurfaceIntuitiveRelationshipsBetweenVariablesBoundedVolume}

Unbounded volume occurs when volume continually increases, allowing for the model to ask for an infinite amount of work to be done. This presents a serious issue because this is not possible for a lifter to complete, and is especially problematic considering the purpose of this entire chapter is to establish what is possible and what is not possible. As such, volume will need to be bounded. This could mean that volume reaches a global maximum or, preferably to allow for property 1.b to be true, volume should reach a plateau. Either way, once volume is bounded a maximum achievable volume can be established.

\subsection{Property 6: 1RM Estimations}
\label{sec:PotentialSurfaceIntuitiveRelationshipsBetweenVariables1RMEstimations}

Being able to accurately predict a lifters 1RM is extremely important. There are several reasons knowing a lifters current 1RM is important. First is because of the models dependence on knowing the lifters current 1RM which will be discussed in section \ref{sec:PotentialSurfaceLinearRegressionAndTimeSeriesProblems}. Secondly, it is an important piece of information for tracking progress. The entire goal of a powerlifter is to increase there 1RM over time, which makes it a good measure of progress. Lastly, it is simply just good information for the lifter to know. If the prediction is good enough then it can be used to tell a lifter what weight is safe to load when attempting a new 1RM. This estimation will only be as useful as it is accurate, so, before moving forward an acceptable error needs to be established. If the predicted intensity is, on average, within $1\%$ of the actual intensity then the model will be considered 'accurate', although greater accuracy could not hurt.

By this point it should be obvious the crux of the problem in question will require finding a lifters volume tolerance across intensity and effort, as well as the lifters volume skew between sets and reps. In order to solve this problem, a surface will need to be found that exhibits the appropriate behaviors and then that surface will need to be fitted to the data. These surfaces will be called \textit{potential surfaces}, because they represent a lifters potential performance. It is important to conceptualize that every combination of sets, reps, weight, and effort defined by a potential surface is theoretically possible for a lifter to complete. This is the foundation this entire book will build off of.


\section{Linear Regression and Time Series Problems}
\label{sec:PotentialSurfaceLinearRegressionAndTimeSeriesProblems}

Before moving forward with the goal of finding a surface of best fit using the data set and linear regression, the limits of linear regression need to be considered. Looking at the data set, it should be clear that it is a time series: the data points are collected over time, and data points nearer in time have greater importance than data points farther away in time. Linear regression assumes that all the data points are uncorrelated, which is clearly untrue with time series data. To reconcile the differences between working with linear regression and a time series data set, two things can be done.

The first step that can be taken is to limit the time frame that linear regression is run over. Sets, reps, volume, and the volume skew can all vary through time, but not necessarily in direct linear relation to time. As an example of this, consider seasonal training, which some powerlifters choose to employ. In the off season intensities will drop, volume will increase, and the emphasis on training may shift to hypertrophy over strength. In the on season intensities will increase, volume will necessarily decrease, and the emphasis on training will shift to strength over hypertrophy. This will result in shifts in the number of sets and reps being done over time. To capture these changes over time several techniques can be used. The exact equations that determine what data points are included and to what extent they are considered will be discussed more in chapter \ref{sec:TimeFrame}. That being said, some notation is required to continue. The general notation used to show which data points will be used when performing linear regression is shown in equation \ref{eq:TimeFrame}. The exact length of the time frame, $t_f$, along with any other specifics, will be discussed in chapter \ref{sec:TimeFrame}. Throughout the rest of this book equation \ref{eq:TimeFrame} will be written short hand as $t_i\in \{ T \}$ In an effort to save space.

\begin{equation}
    \label{eq:TimeFrame}
    %t_t\le t_i\le t_t+t_f
    t_i \in\{ t_t, t_t+1,\dots,t_t+t_f \}
    \equiv t_i \in \{ T \}
\end{equation}
\centerline{where}
\begin{equation*}
    \begin{split}
        t_i &\equiv \text{The time component of an arbitrary data point }i \\
        t_t &\equiv \text{An arbitrary target time} \\
        t_f &\equiv \text{The time frame linear regression will be run over} \\
        T & \equiv \text{A shorthand notation representing the set of all allowed times}
    \end{split}
\end{equation*}

The second step that can be taken is to remove weights correlation with time. Weight correlating with time makes intuitive sense, as the goal of a powerlifter is to continually increase weight over time. As an example, say a lifter starts out with a 1RM of $300$ lbs on squat, but through training is able to increase that weight over time to $800$ lbs. It should be obvious from the example that weight has a positive correlation with time. To fix this, weight will be replaced by intensity. This will remove the correlation with time because intensity is calculated using an exercises 1RM, which itself changes with time, canceling out any effects from an increase in weight through time. To show this the previous example will be used again. Despite weight increasing over time, intensity would generally be limited to the range $0\le I\le 1$ because the lifter would set new 1RM's over time, thereby forcing the intensities to be generally less than $1$.

Having an accurate tracking of a lifters 1RM over time becomes an important dependence for the model now that weight is replaced by intensity. For a powerlifter, this is generally not a problem as a lifts 1RM is usually tested once a macrocycle. Problems arise however when things like injuries, or other unplanned problems occur. During these situations a lifters 1RM is not known and can change rapidly or extremely slowly over time. A solution to problems like this will be explored in sections \ref{sec:TimeFrameDynamicTimeFrameAnalysis} and \ref{sec:TimeFrameInjuriesAndChanges}.


\section{The Effort-Fatigue Model}
\label{sec:PotentialSurfaceEffortFatigueModel}

While exploring the intuitive relationships between variables a large amount of emphasis was placed on volume, with most variables being discussed in relation to volume. Despite this, this section is more concerned with how the surface is going to be defined, which will not involve directly involve volume. Instead, the general idea of how the surfaces presented in the next couple sections were conceptualized will be discussed. Within each surfaces chapter, there will be a discussion about how the surface abides by the intuitive behaviors, if at all.

With that, the starting point of every surface is amusingly simple, and can be summed up in the following phrase: Performance is effort diminished by fatigue. To be more specific, performance is equal to total effort diminished by total fatigue. It is tempting to say that, for a powerlifter, performance is simply the weight that is lifted, making $P=w$. However, remembering that these surfaces will eventually be fitted to the data using linear regression and after the discussion in section \ref{sec:PotentialSurfaceLinearRegressionAndTimeSeriesProblems} it should be clear that intensity should be used in the place of weight to measure performance, making $P=I$. Equation \ref{eq:PerformanceProxy} shows an approximate form for the effort-fatigue model.

\begin{equation}
	\label{eq:PerformanceProxy}
	\begin{split}
			P &\hat{=} E_{tot}\hat{-}F_{tot} \\
			I &\hat{=} E_{tot}\hat{-}F_{tot}
	\end{split}
\end{equation}

The $\hat{=}$ symbol in this case means "something like". It is chosen to mean this over something like the $\approx$ symbol because these equations are approximations in the loosest sense. The equations in this section merely form a rough template for later chapters to fully realize. The $\hat{-}$ symbol means "diminished by". The true mathematical interpretation of the $\hat{-}$ symbol is left for future chapters to define.

After performance, total effort is the next easiest term to understand as it directly correlates to how effort was discussed in section \ref{sec:CommonTermsSection}. The only effort that can be applied is by the lifter themselves, no other external factor can 'add effort' to a lifters workout. \footnote{An argument could be made for pre-workout, sugar rushes, a lifter 'getting in there head', or adrenaline dumps 'adding effort', but these are only tools to help a lifter to exert more effort, not tools to generate more effort in and of themselves. Even with these tools, at the end of the day, the effort still has to come from the lifter themselves.} Given this, $E_{tot}$ can be nicely summed up in equation \ref{eq:TotalEffort}. Because there is only one source of effort, a true equality sign can be used. $\epsilon_1$ is a constant that will need to be found after performing linear regression on whatever concrete representation of the surface the next few chapters decide upon.

\begin{equation}
	\label{eq:TotalEffort}
	E_{tot}=\epsilon_1 E
\end{equation}

Total fatigue is more complicated, as there are several different types of fatigue that contribute to it's total. Given the four different types of fatigue listed in section \ref{sec:ModifiedFatigueCategories}, total fatigue can be represented as follows. Total fatigue is more open to intrepretation than total effort, hence the $\hat{=}$ symbol.

\begin{equation*}
	F_{tot} \hat{=} \epsilon_2 F_l+\epsilon_3 F_w+\epsilon_4 F_e+\epsilon_5 F_s
\end{equation*}

Section \ref{sec:AugmentedDataSet} introduced indexes to represent inter-workout and inter-exercise fatigue. These data points can be directly used in the above equation, however as discussed in section \ref{sec:AugmentedDataSet} the index values only represent the fatigue present before the exercise is performed. The fatigue generated from the current exercise needs to be considered to avoid scenarios where the lifter fails a lift due to being too fatigued to make it through the entire exercise. This is where inter-set fatigue matters. Inter-set fatigue will increase in proportion to to total number of reps done. Two additional terms can be added to measure the fatigue generated from sets and reps individually.

\begin{equation*}
	F_{tot} \hat{=} \epsilon_2 F_l+\epsilon_3 F_w+\epsilon_4 F_e+\epsilon_5 sr+\epsilon_6 s+\epsilon_7 r
\end{equation*}

Finally, the approximate form of the effort-fatigue model can be defined, and is shown in equation \ref{eq:PotentialSurfaceEquation}. Again, this may seem like a finished equation, but remember this is just a template. The surfaces in the next several chapters will create final equations to concretely realize the ideas of the effort-fatigue model, apply constraints, and enforce desired behavior. These equations may take very different forms, but the general idea of the effort-fatigue model still stands behind them.

\begin{equation}
	\label{eq:PotentialSurfaceEquation}
	\begin{split}
		I & \hat{=} E_{tot}\hat{-}F_{tot} \\
		I & \hat{=} \epsilon_1 E\hat{-}\left( 
			\epsilon_2 F_l+\epsilon_3 F_w+\epsilon_4 F_e+\epsilon_5 sr+\epsilon_6 s+\epsilon_7 r
		\right)
	\end{split}
\end{equation}

Before going to the next chapters, it is worth mentioning that latent fatigue will be discussed in chapter \ref{sec:}. It has it's rightful place in the potential surface, but it's implications are so broad that they need a chapter of there own. Until chapter \ref{sec:} the effects from latent fatigue will not be considered. For now, think of it as though $\epsilon_2$ were set to $0$. 


\chapter{Surface 1: The Basic Surface}
\label{sec:PotentialSurfaceTheBasicSurface}

The first surface to explore takes the effort-fatigue model in and makes as few changes as possible, treating the $\hat{-}$ symbol to mean literal subtraction and the $\hat{=}$ to mean literal equality. The first change it makes is to center the surface at $(s=1,r=2,I=\epsilon)$ so that the peak in intensity occurs with one set of one rep, matching the lifters 1RM. The second change is adding an error term, represented by $\epsilon$. This is a necessary term to add to perform linear regression. The next, and most substantial change, is to total fatigue. The discussion in section \ref{sec:PotentialSurfaceIntuitiveRelationshipsBetweenVariables} needs to be considered. One of the main takeaways from that section is that volume is limited by effort at high intensities. If the total fatigue template presented by the effort-fatigue model is taken as a literal equation, volume will not have the correct behavior as intensity approaches $100$\%. Namely, there will not be diminishing returns in volume as intensity increases because all the relationships with sets and reps are either linear or a linear combination of both of them. To correct for this, the terms containing sets and reps will be squared, creating diminishing returns to volume with respect to intensity that can be adjusted through the constant in front of each term. \footnote{Yes, there are more flexible ways to do this, but using those ways would make it so linear regression could not be performed due to the multiplicative combinations of constants.} With these changes, the total fatigue equation for the basic surface is shown below. 

\begin{equation*}
	F_{tot} = \epsilon_2 F_l+\epsilon_3 F_w+\epsilon_4 F_e+\epsilon_5 (s-1)^2(r-1)^2+\epsilon_6 (s-1)^2+\epsilon_7 (r-1)^2
\end{equation*}

The final form of the basic surface is shown in equation \ref{eq:BasicSurfaceEquation}. Several constraints are applied to the constants, the reasoning of which will be discussed later in the chapter, but the general idea behind them is to limit the surface to behavior that makes sense for the problem at hand. This equation attempts to fully realize all of the intuitive relationships discussed in section \ref{sec:PotentialSurfaceIntuitiveRelationshipsBetweenVariables}. Which intuitive relationships it fulfills and which it does not will be the topics of sections \ref{sec:PotentialSurfaceAnalysisOfProperty2}-\ref{sec:PotentialSurfaceAnalysisOfProperty4}.

\begin{minipage}{\textwidth}
	\begin{equation}
		\label{eq:BasicSurfaceEquation}
		\begin{split}
			I & =\epsilon+\epsilon_1 E-\left( 
				\epsilon_2 F_l+\epsilon_3 F_w+\epsilon_4 F_e+\epsilon_5 (s-1)^2(r-1)^2+\epsilon_6 (s-1)^2+\epsilon_7 (r-1)^2
			\right)
		\end{split}
	\end{equation}
	\centerline{where}
	\begin{equation*}
	    \begin{split}
	        E & \in \{ 1,1.5,2,2.5, \dots ,10 \} \\
	        \epsilon, \epsilon_1, \epsilon_2, \epsilon_3, \epsilon_4, \epsilon_5,\epsilon_6,\epsilon_7 & > 0 \\
	    \end{split}
	\end{equation*}
\end{minipage}

To run linear regression the error equation is needed, which is shown below.

\begin{equation*}
    E_{rr}=\sum_{
            \substack{i=0\\ t_i\in \{ T \}}
        }^n \left(
        I_i
        -\epsilon
        -\epsilon_1 E_i
        +\epsilon_2 F_{l,i}
        +\epsilon_3 F_{w,i}
        +\epsilon_4 F_{e,i}
        +\epsilon_5 (s_i-1)^2(r_i-1)^2
        +\epsilon_6 (s_i-1)^2
        +\epsilon_7 (r_i-1)^2
    \right)^2
\end{equation*}

To minimize the error, the partial derivatives of each unknown constant need to be found and each one set equal to zero. An example with $\epsilon$ is shown below.

\begin{equation*}
    \begin{split}
        \frac{\partial E_{rr}}{\partial \epsilon}=
        \frac{\partial}{\partial \epsilon}\sum_{
                \substack{i=0\\ t_i\in \{ T \}}
            }^n \left(
            I_i
            -\epsilon
            -\epsilon_1 E_i
            +\epsilon_2 F_{l,i}
            +\dots
            %+\epsilon_3 F_{w,i}
            %+\epsilon_4 F_{e,i}
            %+\epsilon_5 (s_i-1)^2(r_i-1)^2
            +\epsilon_6 (s_i-1)^2
            +\epsilon_7 (r_i-1)^2
        \right)^2&=0\\
        -2\sum_{
                \substack{i=0\\ t_i\in \{ T \}}
            }^n \left(
            I_i
            -\epsilon
            -\epsilon_1 E_i
            +\epsilon_2 F_{l,i}
            +\dots
            %+\epsilon_3 F_{w,i}
            %+\epsilon_4 F_{e,i}
            %+\epsilon_5(s_i-1)^2(r_i-1)^2
            +\epsilon_6(s_i-1)^2
            +\epsilon_7(r_i-1)^2
        \right)&=0\\
        \epsilon \sum_{\substack{i=0\\ t_i\in \{ T \}}}^n 1 
        +\epsilon_1 \sum_{\substack{i=0\\ t_i\in \{ T \}}} E_i
        -\epsilon_2 \sum_{\substack{i=0\\ t_i\in \{ T \}}} F_{l,i}
        -\dots
        %-\epsilon_3 \sum_{\substack{i=0\\ t_i\in \{ T \}}} F_{w,i}
        %-\epsilon_4 \sum_{\substack{i=0\\ t_i\in \{ T \}}} F_{e,i}
        %-\epsilon_5 \sum_{\substack{i=0\\ t_i\in \{ T \}}}^n (s_i-1)^2(r_i-1)^2
        -\epsilon_6 \sum_{\substack{i=0\\ t_i\in \{ T \}}}^n (s_i-1)^2
        -\epsilon_7 \sum_{\substack{i=0\\ t_i\in \{ T \}}}^n (r_i-1)^2
        &=
        \sum_{\substack{i=0\\ t_i\in \{ T \}}}^n I_i\\
    \end{split}
\end{equation*}

After a similar process is carried out for each unknown constant, equation \ref{eq:LinearRegMatrixEquation} can be used, and the unknown constants can be solved for.

\begin{equation}
    \label{eq:LinearRegMatrixEquation}
	\left[
    \begin{matrix}
        \sum_{i=0}^n 1 &
        \dots &
        -\sum_{\substack{i=0\\ t_i\in \{ T \}}}^n (r_i-1)^2 \\

        \sum_{\substack{i=0\\ t_i\in \{ T \}}}^n  E_i &
        \dots &
        -\sum_{\substack{i=0\\ t_i\in \{ T \}}}^n E_i (r_i-1)^2\\

        \vdots &
        \ddots &
        \vdots \\
        
        \sum_{\substack{i=0\\ t_i\in \{ T \}}}^n (r_i-1)^2 &
        \dots &
        -\sum_{\substack{i=0\\ t_i\in \{ T \}}}^n (r_i-1)^4
    \end{matrix}
    \right]
	\left[
    \begin{matrix}
        \epsilon \\ \epsilon_1 \\ \vdots \\ \epsilon_7
    \end{matrix}
    \right]
    =
	\left[
    \begin{matrix}
        \sum_{\substack{i=0\\ t_i\in \{ T \}}}^n I_i \\
        \sum_{\substack{i=0\\ t_i\in \{ T \}}}^n I_i E_i \\
        \vdots \\
        \sum_{\substack{i=0\\ t_i\in \{ T \}}}^n I_i(r_i-1)^2 
    \end{matrix}
    \right]
\end{equation}

With equations \ref{eq:BasicSurfaceEquation} and \ref{eq:LinearRegMatrixEquation}, this surface is fully defined. After fitting the surface to the data discussed in chapter \ref{sec:DataSection}, graphs similar to the ones shown in figure \ref{fig:SquatPotentialSurfaceAcrossEffort} can be created. Many of the relationships discussed in section \ref{sec:PotentialSurfaceIntuitiveRelationshipsBetweenVariables} are concerned with volume. To give some intuition for volume before moving on, equation \ref{eq:IntensitySubedInVolume} is the result of solving the potential surface for $I$ and substituting it in equation \ref{eq:IntensityBasedVolumeEquation}. The graphs shown in figure \ref{fig:SquatPotentialSurfaceVolumeAcrossEffort} show the volume surface generated from this potential surface.

\begin{equation}
    \label{eq:IntensitySubedInVolume}
    \begin{split}
    		v = & srI l_{1RM} \\
    			= & l_{1RM} sr \left( 
    			\epsilon+
    			\epsilon_1 E-
    			\epsilon_2 F_l-
    			\epsilon_3 F_w-
    			\epsilon_4 F_e-
    			\epsilon_5(s-1)^2(r-1)^2-
    			\epsilon_6(s-1)^2-
    			\epsilon_7(r-1)^2
    		\right)
    \end{split}
\end{equation}

\begin{figure}[htbp]
    \centering
    \includegraphics[scale=0.55]{images/ch3/PotentialSurface/DualSquat.Effort[5,10].basic.png}
    \caption{The potential surface fitted to squat data at various effort levels. The window and time frame values will be discussed in chapter \ref{sec:TimeFrame}. For now, the $I(1,1)$ and MSE values are the most important.}
    \label{fig:SquatPotentialSurfaceAcrossEffort}
\end{figure}

\begin{figure}[htbp]
    \centering
    \includegraphics[scale=0.55]{images/ch3/Volume/DualSquat.Effort[5,10].basic.png}
    \caption{The volume surface resulting from the potential surface fitted to squat data at various effort levels.}
    \label{fig:SquatPotentialSurfaceVolumeAcrossEffort}
\end{figure}

\section{Analysis of Property 1.a: Volume and Intensity}
\label{sec:PotentialSurfaceAnalysisOfProperty1a}

Property 1.a states that volume should approach the lifters 1RM as intensity increases. To prove this two things need to happen. It needs to be proven that equation \ref{eq:BasicSurfaceEquation} has a single global maximum and no minimum. This maximum should equal the lifters current 1RM. Given that this is the case, it then needs to be proven that equation \ref{eq:IntensitySubedInVolume} has a  minimum at the same point that equation \ref{eq:BaseIntensityEquation} has it's maximum. This minimum also needs to equal the value of equation \ref{eq:BaseIntensityEquation} at its maximum. Given that all of that is true, it can be said that the basic surface exhibits property 1.a because volume has a minimum at the same location intensity has a global maximum, meaning intensity increases from all directions to that point where volume has a value equal to the lifters current 1RM.

To begin, the maximum of equation \ref{eq:BaseIntensityEquation} needs to be found. This will require the gradient of that function, shown below.

\begin{equation*}
	\nabla I =
	\left[
	\begin{matrix}
		\frac{\partial I}{\partial E} \\
		\frac{\partial I}{\partial F_l} \\
		\frac{\partial I}{\partial F_w} \\
		\frac{\partial I}{\partial F_e} \\ 
		\frac{\partial I}{\partial s} \\
		\frac{\partial I}{\partial r} \\
	\end{matrix}
	\right] =
	\left[
	\begin{matrix}
		\epsilon_1 \\
		-\epsilon_2 \\
		-\epsilon_3 \\
		-\epsilon_4 \\
		-2(s-1)\left( \epsilon_5(r-1)^2+\epsilon_6 \right) \\
		-2(r-1)\left( \epsilon_5(s-1)^2+\epsilon_7 \right) \\
	\end{matrix}
	\right]
\end{equation*}

To find critical points where a maximum could occur the gradient will be set to zero, $\nabla I=0$, and all the variables can be solved for. Looking at the above equation, there are only two variables remaining, $s$ and $r$. This means that all the variables $E$,$F_l$,$F_w$, and $F_e$ can be any value and the location of the maximum will remain unchanged. Given that only the last two equations of $\nabla I$ effect the result, the problem of finding critical points is essentially rendered down to a two dimensional problem. After some algebraic manipulation, it can be shown that the only critical point is $(s=1,r=1)$.

Now, given the critical point at $(s=1,r=1)$ it needs to be proven to be a maximum. The simplification to a two dimensional problem will be used here, and the standard determinant equation can be used to determine concavity.

\begin{equation*}
	\begin{split}
		\partial_{ss}I & = -2\left( 
			\epsilon_5(r-1)^2+\epsilon_6 
		\right) \\
		\partial_{rr}I & = -2\left( 
			\epsilon_5(s-1)^2+\epsilon_7 
		\right) \\
		\partial_{sr}I & = -4 \epsilon_5 (s-1)(r-1) \\
		D(s=1,r=1) & = \partial_{ss}I(1,1) \partial_{rr}I(1,1)-\left(
			\partial_{sr} I(1,1)
		\right)^2 = 4\epsilon_6 \epsilon_7
	\end{split}
\end{equation*}

Given the constraints on equation \ref{eq:BaseIntensityEquation} it can be said that $D(s=1,r=1)>0$. Given this and the fact that $\partial_{ss}I(s=1,r=1)=-2\epsilon_6$, which is $<0$, it can be stated that the critical point $(s=1,r=1)$ is a maximum. This maximum needs to be shown to equal the lifters 1RM. To obtain this, equation \ref{eq:BaseIntensityEquation} is simply evaluated at $(s=1,r=1)$ and scaled by $l_{1RM}$ to get weight instead of intensity. The result is shown in equation \ref{eq:BasicSurface1RMPred}, which represents what the model thinks the lifters 1RM is. This is an important concept, especially because of the discussion in section \ref{sec:PotentialSurfaceLinearRegressionAndTimeSeriesProblems}, and will be returned to in section \ref{sec:PotentialSurface1RMEstimations}.

\begin{equation}
	\label{eq:BasicSurface1RMPred}
	l_{1RM} I(s=1,r=1)=l_{1RM} \left(
		\epsilon+
    		\epsilon_1 E-
    		\epsilon_2 F_l-
    		\epsilon_3 F_w-
    		\epsilon_4 F_e
    	\right)
\end{equation}

With that, the first part of the proof is complete. The intensity equation has a single max that represents the lifters current 1RM. The second part of the proof relies on finding a minimum of the volume surface. To do this, the gradient of equation \ref{eq:IntensitySubedInVolume} is needed.

\begin{equation*}
	\begin{split}
		\nabla v = & \left[\begin{matrix}
			\frac{\partial v}{\partial E} \\
			\frac{\partial v}{\partial F_l} \\
			\frac{\partial v}{\partial F_w} \\
			\frac{\partial v}{\partial F_e} \\
			\frac{\partial v}{\partial s} \\
			\frac{\partial v}{\partial r} \\
		\end{matrix}\right]=
		\left[\begin{matrix}
			l_{1RM} \epsilon_1 sr \\
			-l_{1RM} \epsilon_2 sr \\
			-l_{1RM} \epsilon_3 sr \\
			-l_{1RM} \epsilon_4 sr \\
			\partial_s v \\
			\partial_r v\\
		\end{matrix}\right] \\
		\partial_s v = & l_{1RM} \left(
			\epsilon r+
			\epsilon_1 Er-
			\epsilon_2 F_l r-
			\epsilon_3 F_w r-
			\epsilon_4 F_e r-
			\epsilon_7 r(r-1)^2	-
			\left( 
				\epsilon_5 r(r-1)^2+\epsilon_6 r
			\right)	
			(3s^2-4s+1)
		\right) \\
		\partial_r v = & l_{1RM} \left(
			\epsilon s+
			\epsilon_1 Es-
			\epsilon_2 F_l s-
			\epsilon_3 F_w s-
			\epsilon_4 F_e s-
			\epsilon_6 s(s-1)^2	-
			\left( 
				\epsilon_5 r(r-1)^2+\epsilon_7 s
			\right)	
			(3r^2-4r+1)
		\right) \\
	\end{split}
\end{equation*}

This time the gradient does not remove any variables, making the simplification that was previously performed unusable. However, looking at the first four equations of the gradient in context of the last two it is clear that the only critical point occurs at $(s=0,r=0)$. However, looking at figures \ref{fig:SquatPotentialSurfaceVolumeAcrossEffort} there appears to be another critical point that represents a maximum. The reason for that is because those graphs have constant values for $E$, $F_l$, $F_w$, and $F_e$. This removes the first four equations of the gradient and as such makes room for another critical point. Given the terms with the $E$, $F_l$, $F_w$, and $F_e$ variables are linear it should come as no surprise that those variables prevent a maximum from occurring, allowing infinite growth in relation to any of the variables.

Going back to the critical point, it needs to be be proven that it is a minimum. However, without the previous simplification the determinant of the Jacobian matrix would be needed to specify if this critical point is a minimum or maximum. Solving that analytically is near impossible, so it will have to suffice looking at the graphs of figure \ref{fig:SquatPotentialSurfaceVolumeAcrossEffort} for proof that this critical point is a minimum. 

Now, this critical point does not match the critical point found from the intensity equation. This presents a problem because unless the points are the same the whole proof falls apart. The saving nuance comes from the domain of the problem, with both $s$ and $r$ being restricted to being $\ge 1$ in section \ref{sec:UnitsOfMeasurement}. This restriction means the minimum in relation to this problem is $(s=1,r=1)$, matching the maximum of equation \ref{eq:BasicSurfaceEquation}. Given this, all that is left to prove is that volume at this critical point is equal to the lifters current 1RM found previously, which is shown in the equation below. With that, it is proven that this surface properly exhibits property 1.a.

\begin{equation}
	v(s=1,r=1)=l_{1RM} \left(
    			\epsilon+
    			\epsilon_1 E-
    			\epsilon_2 F_l-
    			\epsilon_3 F_w-
    			\epsilon_4 F_e 
    	\right) = I(s=1,r=1)
\end{equation}

\section{Analysis of Property 1.b: Volume and Intensity}
\label{sec:PotentialSurfaceAnalysisOfProperty1b}

Property 1.b states that as intensity reaches $0\%$ volume should approach a plateau. However, based on the discussion from the previous section as well as from looking at the graphs in figure \ref{fig:SquatPotentialSurfaceVolumeAcrossEffort} it is already clear that property 1.b is not exhibited by this surface. The previous section found that if $E$, $F_l$, $F_w$, and $F_e$ are considered to be constant then a maximum can occur in the volume surface. Having a maximum removes any possibility of volume reaching a plateau. This is a clear problem with this surface, but despite this the surfaces shown in figure \ref{fig:SquatPotentialSurfaceAcrossEffort} seem to have appropriate behavior and have reasonable mean square errors (MSEs). This is likely because the data collected does not have many points below $50\%$ intensity. The region from $0\%$ to $50\%$ is where the plateau would really start to take effect, with volume above $50\%$ intensity being limited by effort. This means that for a powerlifter, this surface may still be able to produce reasonable results as long as intensity stays above $50\%$ for the vast majority of lifts. For any sport that does push volume at lower intensities, such as crossfit, this surface would not produce good results at all.

\section{Analysis of Property 2: Volume and Effort}
\label{sec:PotentialSurfaceAnalysisOfProperty2}

It needs to be shown that volume increases with increased effort and decreases with decreased effort. As such, the changes in $v$ relating to changes in $E$ are desired, requiring the partial derivative of equation \ref{eq:IntensitySubedInVolume} with respect to $E$.

\begin{equation*}
    \begin{split}
    		\frac{\partial v}{\partial E} & =
    		\frac{\partial}{\partial E} l_{1RM}sr\left( 
    			\epsilon+
    			\epsilon_1 E-
    			\epsilon_2 F_l-
    			\epsilon_3 F_w-
    			\epsilon_4 F_e-
    			\epsilon_5(s-1)^2(r-1)^2-
    			\epsilon_6(s-1)^2-
    			\epsilon_7(r-1)^2
    		\right) \\
    		& =\epsilon_1 l_{1RM} sr
    \end{split}
\end{equation*}

If $\partial_{E}v$ is always $>0$, then it can be said the function is strictly increasing, meaning volume strictly increases with increased effort. The following inequalities are given in section \ref{sec:UnitsOfMeasurement}.

\begin{equation*}
    \begin{split}
        s \ge & 1 \\
        r \in & \{ \mathbb{R}\ge 1 \} \\
        w > & 0
    \end{split}
\end{equation*}

Given that $l_{1RM}$ is a weight, the following conclusion can be made from the above inequalities.

\begin{equation*}
    \epsilon_1 l_{1RM} sr> 0 \textbf{ iff } \epsilon_1> 0
\end{equation*}

Therefore, it can be concluded that volume increases with increased effort if and only if $\epsilon_1> 0$. A similar argument can be made that proves volume decreases with decreased effort if and only if $\epsilon_1> 0$. Of course, this behavior will need to be ensured, so a constraint is be placed on $\epsilon_1$. This constraint is listed in equation \ref{eq:PotentialSurfaceEquation}.

\section{Analysis of Property 3: Volume and Fatigue}
\label{sec:PotentialSurfaceAnalysisOfProperty3}

For property 2, it needs to be shown that volume decreases with increased fatigue. Similar to the previous section, partial derivatives are required. A partial derivative is needed for every fatigue term. Luckily all the fatigue terms are similar in nature, so for brevity only one will be analyzed here and it should be known that the other fatigue terms will follow the same behaviors. To begin, the partial derivative of equation \ref{eq:IntensitySubedInVolume} with respect to $F_l$ is required.

\begin{equation*}
    \begin{split}
    		\frac{\partial v}{\partial F_l} & =
    		\frac{\partial}{\partial F_l} l_{1RM} sr\left( 
    			\epsilon+
    			\epsilon_1 E-
    			\epsilon_2 F_l-
    			\epsilon_3 F_w-
    			\epsilon_4 F_e-
    			\epsilon_5(s-1)^2(r-1)^2-
    			\epsilon_6(s-1)^2-
    			\epsilon_7(r-1)^2
    		\right) \\
    		& = -\epsilon_2 l_{1RM} sr
    \end{split}
\end{equation*}

This time, if $\partial_{F_l}v$ is always $<0$ then it can be said the function is strictly decreasing and volume decreases with increased fatigue. The same inequalities that were presented in previous section can be used here and the following statement can be made.

\begin{equation*}
    -\epsilon_2 l_{1RM} sr< 0 \textbf{ iff } \epsilon_2> 0
\end{equation*}

This proves that volume decreases with fatigue if and only if $\epsilon_2>0$. Just like the previous section, a similar argument can be made to prove that volume increases with decreased fatigue. This constraint will need to be applied to $\epsilon_2$, $\epsilon_3$, $\epsilon_4$, $\epsilon_5$, and $\epsilon_7$ as they are all part of $F_{tot}$. These constraints are listed in equation \ref{eq:PotentialSurfaceEquation}. The next section should offer a break from the monotony of this section.

\section{Analysis of Property 4: Sets and Reps}
\label{sec:PotentialSurfaceAnalysisOfProperty4}

Property 4 is concerned with how sets and reps change, mainly with regard to volume skew. As such, it needs to be shown that the model is capable of determining volume skew. An initial look at the potential surface can't hurt. Looking at the potential surface equation the only place where differences between $s$ and $r$ can arise are with the $\epsilon_6$ and $\epsilon_7$ terms, so it makes sense to reason that ratio of those constants will somehow measure volume skew. Exactly how they measure volume skew will need to be determined by calculating the volume underneath the volume surface on either side of the plane $s=r$. Once that is done, the ratio of total volume skewing towards sets and reps respectively can be calculated. 

To measure the total volume on either side of the plane $s=r$ several double integrals will need to be calculated. Given the non-concave nature of the volume surface, calculating the total volume on either side of the $s=r$ plane will require the volume surface to be broken up into $4$ regions. These regions are:

\begin{itemize}
	\item $1\le r\le s$ and $0\le s \le S_d$ where $S_d$ is the point where the volume surface intersects the line $s=r$ on the $sr$ plane.
	\item $1\le s\le r$ and $0\le r \le R_d$ where $R_d$ is the point where the volume surface intersects the line $s=r$ on the $sr$ plane.
	\item $1\le r\le S_t$ and $S_d \le s\le S_e$ where $S_t$ is the intersection of the volume surface on the $sr$ plane solved for $s$ and $S_e$ is the the point where the volume surface intersects the $s=1$ line on the $sr$ plane.
	\item $1\le s\le R_t$ and $R_d \le r\le R_e$ where $R_t$ is the intersection of the volume surface on the $sr$ plane solved for $r$ and $R_e$ is the the point where the volume surface intersects the $r=1$ line on the $sr$ plane.
\end{itemize}

Note that due to the nature of the line $s=r$, $S_d=R_d$. In integral form, the equation will take the form shown in equation \ref{eq:BasicSurfaceVolumeSkew}. Note that this equation creates a fraction of sets over reps. The inverse of equation \ref{eq:BasicSurfaceVolumeSkew} would represent reps over sets. Either way would work as long as it is known which form is used, so, to be clear, the form of sets over reps will be used for the rest of this section.

\begin{minipage}{\textwidth}
	\begin{equation}
	   \label{eq:BasicSurfaceVolumeSkew}
	   v_s=\frac{
	   	\int_{1}^{S_d}\int_{1}^{s}v\;drds+
	   	\int_{S_d}^{S_e}\int_{1}^{R_t}v\;drds
	   }{
	   	\int_{1}^{R_d}\int_{1}^{r}v\;dsdr+
	   	\int_{R_d}^{R_e}\int_{1}^{S_t}v\;dsdr
	   }
	\end{equation}
	\centerline{where}
	\begin{equation*}
		\begin{split}
			v & \equiv \text{ equation \ref{eq:IntensitySubedInVolume}} \\
			S_d=R_d & =\left(
				\frac{
	   				-\epsilon_6-\epsilon_7+\sqrt{
	   					(\epsilon_6+\epsilon_7)^2+
	   					4\epsilon_5(
	   						\epsilon+
	   						\epsilon_1E-
	   						\epsilon_2F_l-
	   						\epsilon_3F_w-
	   						\epsilon_4F_e
	   					)
	   				}
	   			}{
	   				2\epsilon_5
			   	}\right)^{\frac{1}{2}} +1 \\
			S_e & =\left(
				\frac{
					\epsilon+
					\epsilon_1E-
					\epsilon_2F_l-
					\epsilon_3F_w-
					\epsilon_4F_e
				}{
					\epsilon_6
				}
			\right)^{\frac{1}{2}}+1 \\
			R_e & =\left(
				\frac{
					\epsilon+
					\epsilon_1E-
					\epsilon_2F_l-
					\epsilon_3F_w-
					\epsilon_4F_e
				}{
					\epsilon_7
				}
			\right)^{\frac{1}{2}}+1 \\
			S_t & =\left(
				\frac{
					\epsilon+
					\epsilon_1E-
					\epsilon_2F_l-
					\epsilon_3F_w-
					\epsilon_4F_e-
					\epsilon_7(r-1)^2		
				}{
					\epsilon_5(r-1)^2+
					\epsilon_6
				}
			\right)^{\frac{1}{2}}+1 \\
			R_t & =\left(
				\frac{
					\epsilon+
					\epsilon_1E-
					\epsilon_2F_l-
					\epsilon_3F_w-
					\epsilon_4F_e-
					\epsilon_6(s-1)^2		
				}{
					\epsilon_5(s-1)^2+
					\epsilon_7
				}
			\right)^{\frac{1}{2}}+1 \\
		\end{split}
	\end{equation*}
\end{minipage}

Given this equation, the following statements can be made:

\begin{enumerate}
    \item If $v_s<1$ then the total volume from reps is greater than the total volume from sets, which implies that the lifter favors reps for the given exercise, or has a volume skew towards reps for the given exercise.
    \item If $v_s>1$ then the total volume from sets is greater than the total volume from reps, which implies that the lifter favors sets for the given exercise, or has a volume skew towards sets for the given exercise.
    \item If $v_s=1$, then the lifter has no volume skew.
    \item If $\epsilon_6=0$ or $\epsilon_7=0$ then a lifter has the maximum possible skew towards sets or reps respectively. \footnote{This idea will be explored more from volumes perspective in section \ref{sec:PotentialSurfaceUnboundedVolume}.}
\end{enumerate}

Several things are of immediate interest from equation \ref{eq:BasicSurfaceVolumeSkew}. The first is whether or not the top and bottom of the fraction are equivalent. If they are equivalent then this means the surface cannot measure volume skew. Luckily they are not equivalent due to the small differences in the placement of the $\epsilon_6$ and $\epsilon_7$ terms in the integral bounds. The first term on the top and bottom of the fraction may look equivalent because the bounds of integration are the same for both terms, but the order of the variables being integrated prevents this, this time because of the difference between the $\epsilon_6$ and $\epsilon_7$ terms in the volume equation. With this, property 4 is proven to be satisfied as the surface can measure volume skew.

The second thing of interest is that this equation is utterly unusable algebraically. It would be quite nice to see the fraction reduce down to some jumble of constants that would be somewhat palatable, but the second term on the top and bottom of the fraction makes this impossible. The first term on the top and bottom of the fraction would be very tedious to compute by hand, but would technically be possible due to the simplicity of the inner integrals bounds. However, in the second term the $R_t$ and $S_t$ equations present a rather nasty upper bound on the inner integral that would have to be integrated after substituting it in the integral of the volume equation. This just results in more of a mess than it is worth trying to untangle. Because of this, an alternate way to measure volume skew would be helpful to have.

The starting point for creating an approximation for volume skew is noticing that, generally, $S_e$ and $R_e$ each increase with the total volume on there respective sides of the $s=r$ plane. As such, the approximation for volume skew can be represented as the ratio of those terms, shown in equation \ref{eq:BasicSurfaceApproxVolumeSkew}. In order to match equation \ref{eq:BasicSurfaceVolumeSkew}, this equation will also create a fraction of sets over reps.

\begin{equation}
	\label{eq:BasicSurfaceApproxVolumeSkew}
	\begin{split}
		v_s \approx \frac{S_e}{R_e} & =\frac{
			\left(
				\frac{
					\epsilon+
					\epsilon_1E-
					\epsilon_2F_l-
					\epsilon_3F_w-
					\epsilon_4F_e
				}{
					\epsilon_6
				}
			\right)^{\frac{1}{2}}+1
		}{
			\left(
				\frac{
					\epsilon+
					\epsilon_1E-
					\epsilon_2F_l-
					\epsilon_3F_w-
					\epsilon_4F_e
				}{
					\epsilon_7
				}
			\right)^{\frac{1}{2}}+1
		} \\
		& = \frac{
			\epsilon_7^{\frac{1}{2}} \left(
				\left(
					\epsilon+
					\epsilon_1E-
					\epsilon_2F_l-
					\epsilon_3F_w-
					\epsilon_4F_e
				\right)^{\frac{1}{2}}+
				\epsilon_6^{\frac{1}{2}}
			\right)		
		}{
			\epsilon_6^{\frac{1}{2}} \left(
				\left(
					\epsilon+
					\epsilon_1E-
					\epsilon_2F_l-
					\epsilon_3F_w-
					\epsilon_4F_e
				\right)^{\frac{1}{2}}+
				\epsilon_7^{\frac{1}{2}}
			\right)		
		}
	\end{split}
\end{equation}

Because equation \ref{eq:BaseVolumeEquation} cannot be algebraically manipulated, the efficacy of this approximation will need to be shown analytically. The graphs in figure \ref{fig:ApproximateVsActualVolumeSkew} show the true volume skew compared to the approximation for varying values of different $\epsilon$ constants. As the graphs show, the approximation holds well enough to be considered a valid approximation. Of note is that when the actual volume skew equals $1$ so does the approximation. This is important because this is a critical point where the direction of the skew changes, so exactly matching this value in the approximation is important. Also of note is that as the approximation strays from $1$ it gets worse, with large $\epsilon_7$ values being particularly troublesome. The reason $\epsilon_7$ values have a greater error relates to the nature of the fraction. Having a skew towards reps constrains volume skew to be between $0\le v_s<1$ whereas a skew towards sets constrains the fraction to be between $1<v_s\le \infty$. Both scenarios need to represent the same range of values, but the skew towards reps has a much smaller range to do so, which also leaves much less space for error. This weakness of the volume skew approximation can also be seen in graph \ref{fig:ApproximateVsActualVolumeSkewOverTime} where volume skew values $>1$ have much greater separation from there approximated values.

\begin{figure}[htb]
    \centering
    \includegraphics[scale=0.55]{images/ch3/ApproxVsActualVolumeSkew.basic.png}
    \caption{Graphs comparing the approximate and actual volume skew over varrying $\epsilon$ values. Note how the approximation follows the same direction of rate of change of the true volume skew, making it a reasonable approximation for this particular use case.}
    \label{fig:ApproximateVsActualVolumeSkew}
\end{figure}
\begin{figure}[htb]
    \centering
    \includegraphics[scale=0.55]{images/ch3/volumeSkewOverTime.png}
    \caption{Graphs comparing the approximate and actual volume skew over time. Note how the volume skew values $>1$ have much greater error.}
    \label{fig:ApproximateVsActualVolumeSkewOverTime}
\end{figure}

Before moving on, there has been so much attention on $\epsilon_6$ and $\epsilon_7$ that $\epsilon_5$ has not been discussed, despite having $s$ and $r$ terms associated with it. The constant $\epsilon_5$ represents the volume that should be possible no matter the magnitude of the volume skew. For example, if $\epsilon_6=0$ and $\epsilon_7>0$ for a given exercise, then the following can be stated about the lifter and the way they perform that particular exercise:

\begin{itemize}
    \item The lifter has a volume skew towards reps.
    \item As the lifter leaves there normal volume skew, they should still be able to perform volume in proportion to $\epsilon_5$ even when $r\to 1$ and the contributions from $\epsilon_7$ diminish.
    \item The volume from $\epsilon_5$ is less than any volume that would have been present if $\epsilon_6>0$ because the relation between $\epsilon_5$ and $\epsilon_6$ is additive.
\end{itemize}

Having this representation for baseline work capacity in the opposing volume skew is important. If it was not present the model would assume when a lifter has a certain volume skew, any volume from the opposing skew would simply not be possible. In reality, this is not the case, some baseline level of volume will always be possible, even in the opposite skew. Do not confuse this with volume tolerance, which represents a maximal volume that a lifter can tolerate rather than a baseline of volume possible in the opposing volume skew.

\section{Analysis of Property 5: (Un)Bounded Volume}
\label{sec:PotentialSurfaceUnboundedVolume}

One way to prove volume is bounded, and hence prove the surface properly exhibits property 5, is to prove that it has a maximum value. Given the discussion in section \ref{sec:PotentialSurfaceAnalysisOfProperty1a} this maximum does appear to be present when $E$, $F_l$, $F_w$, and $F_e$ are all considered to be constant. The question now is if that maximum is always present. Figuring that out will require finding the solution to the $\partial_sv$ and $\partial_rv$ system of equations shown in the gradient of $v$ in section \ref{sec:PotentialSurfaceAnalysisOfProperty1a}. However, analytically solving that system of equations is impossible. To get around this, the problem will be looked at from the perspectives of $\partial_sv$ and $\partial_rv$ individually. As such, the first and second partial derivatives of $v$ with respect to $s$ are shown below. It will be discussed later what changes when the partial derivative with respect to $r$ is chosen instead.

\begin{equation*}
    \partial_s v = l_{1RM} \left(
			\epsilon r+
			\epsilon_1 Er-
			\epsilon_2 F_l r-
			\epsilon_3 F_w r-
			\epsilon_4 F_e r-
			\epsilon_7 r(r-1)^2	-
			\left( 
				\epsilon_5 r(r-1)^2+\epsilon_6 r
			\right)
			(3s^2-4s+1)
		\right)
\end{equation*}
\begin{equation*}
    \partial_{ss}v=-l_{1RM} r
    		\left( 
			\epsilon_5 (r-1)^2+\epsilon_6
		\right)
		(6s-4)
\end{equation*}

A global maximum occurs when $\partial_{ss} v<0$. In order for this to be the case either none of the parenthetical groupings needs to be negative or an even number of the above parenthetical groupings need to be negative. Thr first parenthetical grouping is guarintted to be positive due to the constraints placed on $r$ in section \ref{sec:UnitsOfMeasurement}. The last parenthetical grouping will be $>0$ if $s>\frac{2}{3}$, which is again guaranteed to be true because of the constraints placed on $s$ in section \ref{sec:UnitsOfMeasurement}. This guarantees two of the three parenthetical groupings to be positive, which forces the middle parenthetical grouping to be positive for $\partial_{ss} v$ to be $<0$. The sign of $\epsilon_5$ changes the set of inequalities that define when the middle parenthetical group is $>0$, and hence when a maximum can occur. Table \ref{tab:BoundedVolumeRanges} shows the inequalities and ranges when volume is bounded for various values of $\epsilon_5$ and $\epsilon_6$.

Care has to be taken when $\epsilon_5=0$ or $\epsilon_6=0$, as the behavior of the inequalities is not fully captured by substituting $0$ for $\epsilon_5$ or $\epsilon_6$. A traditional limit cannot be used because inequalities are one sided, and a limit expects a continuous function. Instead, the four scenarios where $\epsilon_5=0$ and $\epsilon_6=0$ will be treated as one sided limits which will be approached from the side where the behavior is already known. By doing this, the sign can be inferred and the one sided nature of the problem. \footnote{The notation used in table \ref{tab:BoundedVolumeRanges} is as follows: given a number $x$, $x^-$ is a smaller number approaching $x$, and $x^+$ is a larger number approaching $x$. This allows for one sided analysis to be completed.} The scenarios where the root is imaginary are also troublesome. However, the same one sided analysis can be extended and used again. Table \ref{tab:BoundedVolumeRanges} shows how the one sided limit works as well as how the ranges over which volume is bounded respond.\footnote{For the purposes of this discussion the constraints placed on $\epsilon_2$-$\epsilon_7$ will be ignored. These constraints will be reconsidered at the end of the discussion.}

\begin{table}[h]
    \centering
    \begin{tabular}{|p{1cm}|p{6cm}|c|}
    		\hline
        Values & Inequalities & Range Where Volume is Bounded \\
        \hline

        $\epsilon_5>0$
        \newline
        $\epsilon_6<0$
        &
        $r> \left(-\frac{\epsilon_6}{\epsilon_5}\right)^\frac{1}{2}+1$
        \newline
        $r< -\left(-\frac{\epsilon_6}{\epsilon_5}\right)^\frac{1}{2}+1$
        &
        $\left( r>\left| \frac{\epsilon_6}{\epsilon_5} \right|^\frac{1}{2}+1 \right) \cup \left( r< -\left| \frac{\epsilon_6}{\epsilon_5} \right|^\frac{1}{2}+1 \right)$
        \\
        \hline
        
        $\epsilon_5>0$
        \newline
        $\epsilon_6=0$
        &
        $r> \left(-\frac{0^-}{\epsilon_5}\right)^\frac{1}{2}+1$
        \newline
        $r< -\left(-\frac{0^-}{\epsilon_5}\right)^\frac{1}{2}+1$
        &
        $\left( r> 1^+ \right) \cup \left( r<1^- \right)$, a.k.a. $r\ne 1$
        \\
        \hline
        
        $\epsilon_5>0$
        \newline
        $\epsilon_6>0$
        &
        $r> (0+\beta i)+1$
        \newline
        $r< (-0-\beta i)+1$
        &
        $(r>1)\cup (r<1)$ a.k.a. All $r$
        \\
        \hline
        
        $\epsilon_5=0$
        \newline
        $\epsilon_6>0$
        &
        $r> \left(-\frac{\epsilon_6}{0^+}\right)^\frac{1}{2}+1=(0+\infty i)+1$
        \newline
        $r< -\left(-\frac{\epsilon_6}{0^+}\right)^\frac{1}{2}+1=(-0-\infty i)+1$
        &
        $(r>1)\cup (r<1)$ a.k.a. All $r$
        \\
        \hline
        
        $\epsilon_5<0$
        \newline
        $\epsilon_6>0$
        &
        $r< \left(-\frac{\epsilon_6}{\epsilon_5}\right)^\frac{1}{2}+1$
        \newline
        $r> -\left(-\frac{\epsilon_6}{\epsilon_5}\right)^\frac{1}{2}+1$
        &
        $-\left| \frac{\epsilon_6}{\epsilon_5} \right|^\frac{1}{2}+1 < r< \left| \frac{\epsilon_6}{\epsilon_5} \right|^\frac{1}{2}+1$
        \\
        \hline
        
        $\epsilon_5<0$
        \newline
        $\epsilon_6=0$
        &
        $r< \left(-\frac{0^+}{\epsilon_5}\right)^\frac{1}{2}+1$
        \newline
        $r> -\left(-\frac{0^+}{\epsilon_5}\right)^\frac{1}{2}+1$
        &
        $1^- < r< 1^+$, a.k.a. Never
        \\
        \hline
        
        $\epsilon_5<0$
        \newline
        $\epsilon_6<0$
        &
        $r< (0+\beta i)+1$
        \newline
        $r> (-0-\beta i)+1$
        &
        $1<r<1$ a.k.a. Never
        \\
        \hline
        
        $\epsilon_5=0$
        \newline
        $\epsilon_6<0$
        &
        $r< \left(-\frac{\epsilon_6}{0^-}\right)^\frac{1}{2}+1=(0+\infty i)+1$
        \newline
        $r> -\left(-\frac{\epsilon_6}{0^-}\right)^\frac{1}{2}+1=(-0-\infty i)+1$
        &
        $1<r<1$ a.k.a. Never
        \\
        \hline
    \end{tabular}
    \caption{Various values for $\epsilon_5$ and $\epsilon_6$ as well as the resulting ranges where volume is bounded.}
    \label{tab:BoundedVolumeRanges}
\end{table}

Given the ranges where a maximum could occur, a critical point is still needs to be found, which will require setting $\partial_sv$ equal to $0$. After algebraic manipulation, equation \ref{eq:VolumeRidgeConstR} is generated.
%Using the quadratic formula and substituting $\alpha_s$ in place of the fraction for ease of writing, the following points are generated as critical points.

%the equation below is generated.
\begin{equation}
    \label{eq:VolumeRidgeConstR}
    3s^2-4s+\left( 1-\frac{
    		\epsilon r+
		\epsilon_1 r-
		\epsilon_2 r-
		\epsilon_3 r-
		\epsilon_4 r-
		\epsilon_7 r(r-1)^2		
	}{
		\epsilon_5(r-1)^2-\epsilon_6
	}
    \right)=0
\end{equation}

 Using the quadratic formula and substituting $\eta_s$ in place of the fraction for ease of writing, the following points are generated as critical points.
 
\begin{equation*}
    s=\frac{2\pm \sqrt{1+3\eta_s}}{3}
\end{equation*}

From here the determinant can be used to show where critical points will occur.

\begin{enumerate}
    \item $\eta_s < -\frac{1}{3}$: No critical points
    \item $\eta_s = -\frac{1}{3}$: Indeterminate critical point
    \item $\eta_s > -\frac{1}{3}$: Two critical points
\end{enumerate}

Given the above scenarios for critical points, only option $(3)$ is of interest. As shown below, if $\eta_s> -\frac{1}{3}$ one critical point is generated on either side of $s=\frac{2}{3}$, the exact $s$ boundary where $\partial_{ss} v$ changes sign. Again, $\partial_{ss} v$ needs to $<0$ for a maximum to be found, requiring the parenthetical relating to $s$ in $\partial_{ss} v$ to be $>0$. This further narrows down the critical points to only include the one that is $>\frac{2}{3}$. It is worth mentioning that this is also the only valid critical point from the context of the problem, with $s$ being constrained to be $>1$ in section \ref{sec:UnitsOfMeasurement}.

\begin{equation*}
    s=\frac{2\pm \sqrt{1+3\left( -\frac{1}{3} \right)^+}}{3}=\frac{2\pm 0^+}{3}=\left( \frac{2}{3} \right)^-, \left( \frac{2}{3} \right)^+
\end{equation*}

Given that $\eta_s>-\frac{1}{3}$, there will be a critical point. This can now be combined with the ranges where volume is bounded as well as the context of the problem to create the following statements:

\begin{enumerate}
    \item Volume is fully bounded for all $r$ when $\epsilon_5,\epsilon_6>0$.
    \item When $\epsilon_5>0$ and $\epsilon_6=0$ volume is fully bounded for all $r\ne 1$.
    \item When $\epsilon_5<0$ and $\epsilon_6=0$ volume is never bounded.
    \item When $\epsilon_5=0$ and $\epsilon_6>0$, volume is fully bounded.
    \item When $\epsilon_5=0$ and $\epsilon_6<0$, volume is never bounded.
    \item When $\epsilon_5>0$ and $\epsilon_6<0$, volume is bounded for $r>\left| \frac{\epsilon_6}{\epsilon_5} \right|^\frac{1}{2}+1$.
    \item When $\epsilon_5<0$ and $\epsilon_6>0$, volume is bounded for $1\le r < \left| \frac{\epsilon_6}{\epsilon_5} \right|^\frac{1}{2}+1$
\end{enumerate}

This analysis only looked at sets, which was an assumption made from the very beginning when the partial derivative of $v$ was taken with respect to $s$. If instead, the partial derivative was taken with respect to $r$, an analogous process would be followed. Due to the symmetry from $\partial_sv$ and $\partial_r v$, the only differences will be $\epsilon_7$ replacing $\epsilon_6$, $\eta_r$ replacing $\eta_s$ and, obviously, $s$ and $r$ switching roles. This creates the following additional statements given that $\eta_r>-\frac{1}{3}$:

\begin{enumerate}
    \setcounter{enumi}{7}
    \item Volume is fully bounded for all $s$ when $\epsilon_5,c>0$.
    \item When $\epsilon_5>0$ and $\epsilon_7=0$ volume is fully bounded for all $s\ne 1$.
    \item When $\epsilon_5<0$ and $\epsilon_7=0$ volume is never bounded.
    \item When $\epsilon_5=0$ and $\epsilon_7>0$, volume is fully bounded.
    \item When $\epsilon_5=0$ and $\epsilon_7<0$, volume is never bounded.
    \item When $\epsilon_5>0$ and $\epsilon_7<0$, volume is bounded for $s>\left| \frac{\epsilon_7}{\epsilon_5} \right|^\frac{1}{2}+1$.
    \item When $\epsilon_5<0$ and $\epsilon_7>0$, volume is bounded for $1\le s < \left| \frac{\epsilon_7}{\epsilon_5} \right|^\frac{1}{2}+1$
\end{enumerate}

%For peace of mind, figure \ref{fig:VolumeConstSRPeak} shows the peak volume identified by the above algebraic analysis. There is a peak for all $s$ and $r$ values, which means that volume is fully bounded. This also matches the statements above as $a=1.6\times 10^{-4}$, $b=2.7\times 10^{-3}$, and $c=2.9\times 10^{-3}$, which are all greater than $0$.
%
%\begin{figure}[h]
%    \centering
%    \includegraphics[width=170mm]{DeadliftVolume/FailedRidgeIdentification-2.png}
%    \caption{The peak identified in volume given constant $s$ and $r$ values. This may look similar to figure \ref{fig:DeadliftIntensityCriticalPointsOnVolume} but they are two different graphs that were generated from two different approaches.}
%    \label{fig:VolumeConstSRPeak}
%\end{figure}

The above statements make sense in isolation, but when considered together things can contradict each other. If $\epsilon_5<0$, $\epsilon_6=0$, and $\epsilon_7>0$, then  statement $(3)$ states volume is never bounded, but statement $(14)$ states volume is bounded for a specific region. This is the result of looking at a three dimensional problem from two dimensions. While this does create some problems, the above statements are still is able to give some intuition for the problem at hand. Revisiting the example stated previously, it is saying that volume from reps is never bounded, and volume from sets is only bounded over a specific region.

It is tempting to say the bounds in the above statements can be used in practice to limit the lifter from doing to much volume, but just because volume is bounded does not mean it is reasonable. Just because it can be proven that volume has a peak does not prove that the peak volume is attainable. This is a limitation of the model that stems from using linear regression to fit a surface to the data. Linear regression will implicitly extrapolate what amount of volume can be done given the data it has. Extrapolation is a known weakness of linear regression, making its predicted volume sometimes inaccurate.

In theory, $\epsilon_5$, $\epsilon_6$ and $\epsilon_7$ can be limited to being $> 0$ so that volume is never unbounded. These constraints match those those generated in section \ref{sec:PotentialSurfaceAnalysisOfProperty3} and are shown in equation \ref{eq:BasicSurfaceEquation}. In practice this could lead to a lot of scenarios where the model cannot be applied, severely hindering it's usability. To fix this the range of $\epsilon_5$, $\epsilon_6$ and $\epsilon_7$ can be expanded to being $\ge 0$. Doing this makes it so volume could only ever be unbounded when $\epsilon_5=0$, $\epsilon_6=0$ or $\epsilon_7=0$. This removes the complexities of defining ranges over which the potential surface has bounded volume. After all this work, it can safely be stated that equation \ref{eq:BasicSurfaceEquation} does not properly exhibit property 5.

\section{Analysis of Property 6: 1RM Estimations}
\label{sec:PotentialSurface1RMEstimations}

Property 6 is concerned with 1RM estimations, with it defining an acceptable boundary of accuracy. Table \ref{tab:BasicSurface1RMEstimations} and figure \ref{fig:ApproximateVsActual1RMIntensity} show the models predicted 1RM vs the actual weight that was lifted. \footnote{The lifter did not have access to any of the models estimations when the 1RM attempts were performed, removing any bias from the data that would have been present if the lifter was told the models estimates.} It should be clear from this table that the model consistently under-predicts the lifters actual 1RM with an error much greater than $1\%$. The model almost never predicts above $100\%$ intensity, which is required for a new 1RM.
%The only exception was for heeled squat on $9/10/2022$ but even then the model significantly under-predicted the lifters actual performance.

\begin{figure}[htb]
    \centering
    \includegraphics[scale=0.55]{images/ch3/PredVsActual1RM.basic.png}
    \caption{Graphs comparing the approximate and actual 1RM intensity values. Note how the model nearly always under predicts the lifters current 1RM. Of note, though not further discussed until much later, is how the model seems to be getting more accurate as time passes.}
    \label{fig:ApproximateVsActual1RMIntensity}
\end{figure}
\begin{table}[p]
    \centering
	\csvautotabular{../data/generatedData/Client1.1RMPred.slidingWindow.basic.csv}
	\caption{A table showing predicted vs actual 1RM intensities. Note that the model consistently under predicts the lifters current 1RM.}
	\label{tab:BasicSurface1RMEstimations}
\end{table}

There are a couple reasons the model could be exhibiting this behavior. The most prominent however is due to a decision that was made all the way back at the beginning of this chapter. When setting up the $F_{tot}$ equation the terms containing $s$ and $r$ were arbitrarily squared. It is important to note that squaring the terms lead to a better model than if they were left linear, but the choice of squaring the terms was arbitrary. By choosing a power of $2$ the surface was automatically constrained to a certain range of behaviors, and one of the behaviors was how quickly the surface reaches its maximum. Seeing that the model is consistently under-predicting the lifters 1RM, this may be indicative that the surface is leveling off too quick and hence is not able to reach the lifters 1RM appropriately. So what should this constant be? Rather than choosing an arbitrary value, it's value needs to be appropriately chosen from the data, just like the other constants in the surface equation.

\section{Summary}

Before moving forward it is worth quickly noting the success and failures of this surface so that it can be compared to the surfaces presented in the next few chapters. Out of all six properties, this surface exhibited four correctly. The fist property that was not properly exhibited was property 1.b because the surface did not reach a plateau representing the lifters volume tolerance. Despite this hindrance, the surface did seem to exhibit reasonable behavior above $50\%$ intensity, which could make it a viable surface to use as long as this limitation is kept in mind.

Other than property 1.b, property 5 and 6 were also not properly exhibited. Property 5 dealt with unbounded volume. Constraints were placed on the surface limit the occurrence of unbounded volume, but there can still be scenarios where $\epsilon_5=0$, $\epsilon_6=0$, or $\epsilon_7=0$ where unbounded volume can occur. Property 6 was concerned with 1RM estimations. It was shown that the surface consistently under-predicts the lifters 1RM. This is problematic for several reasons, but largely because a lifters goal is to increase there 1RM, making accurate 1RM estimations necessary to track progress. Table \ref{tab:BasicSurfacePropertiesAndBehaviorsSummary} summarizes the properties and whether or not the surface properly exhibited there behavior. It is clear this surface is far from perfect. Chapter \ref{sec:PotentialSurfaceTheVolumeBaseSurface} will attempt to address some of the previously outlined issues with an updated surface.

\begin{table}[htb]
    \centering
    \begin{tabular}{|c|c|c|}
    		\hline
    		Property & Properly Exhibited & Not Properly Exhibited \\
    		\hline
    		1.a & \cmark & \\
    		1.b & & \xmark \\  		
    		2 & \cmark & \\
    		3 & \cmark & \\
    		4 & \cmark & \\
    		5 & & \xmark \\
    		6 & & \xmark \\
    		\hline
    \end{tabular}
    \caption{Properties and whether or not they are exhibited by this surface.}
    \label{tab:BasicSurfacePropertiesAndBehaviorsSummary}
\end{table}


\chapter{Surface 2: The Volume Base Surface}
\label{sec:PotentialSurfaceTheVolumeBaseSurface}

The second surface builds off of the previous chapter by trying to address one of the weaknesses of the basic surface. Namely, the basic surface does not exhibit property 1.b. In order for this new surface to properly exhibit property 1.b a less literal interpretation of the effort-fatigue model is required. To accomplish this the $\hat{-}$ symbol is going to be treated as division instead of subtraction. This will be done in order to create asymptotic behavior on the $I$ axis which will represent the lifters volume base. Doing this however creates another issue of dividing by zero when total fatigue is zero, requiring a second change. Another constant will be added, $\epsilon_8$, which will be constrained to being $\ne F_{tot}$ to prevent any scenarios where division by zero would occur. The final change is to square intensity, again to make it so volume will exhibit asymptotic behavior on the $I$ axis.\footnote{Confused as to why this is needed? This will be discussed in section \ref{sec:VolumeBasePotentialSurfaceProperty5}.} Equation \ref{eq:VolumeBaseSurfaceEquation} shows the the volume base equation.
 
\begin{equation}
	I^2 = \frac{E_{tot}}{\epsilon_8+F_{tot}}= \frac{
		\epsilon_1E
	}{
		\epsilon_8+
		\epsilon_2 F_l+
		\epsilon_3 F_w+
		\epsilon_4 F_e+
		\epsilon_5 (s-1)^2(r-1)^2+
		\epsilon_6 (s-1)^2+
		\epsilon_7 (r-1)^2
	}
	\label{eq:VolumeBaseSurfaceEquation}
\end{equation}

It is worth noting that this surface does not contain an error term, $\epsilon$, like the basic surface. If it did the constants in the equation would no longer be separable and linear regression could not be performed. Even as the equation stands, some algebraic manipulation will need to be performed before linear regression can be performed. In order to make equation \ref{eq:VolumeBaseSurfaceEquation} usable with linear regression its inverse will need to be found, and linear regression will be run on the inverse of intensity. Taking the inverse of equation \ref{eq:VolumeBaseSurfaceEquation} results in the equation shown below which linear regression can be run on.

\begin{minipage}{\textwidth}
	\begin{equation*}
		\frac{1}{I^2}=
			\epsilon_{8,N}\frac{1}{E}+
			\epsilon_{2,N}\frac{F_l}{E}+
			\epsilon_{3,N}\frac{F_w}{E}+
			\epsilon_{4,N}\frac{F_e}{E}+
			\epsilon_{5,N}\frac{(s-1)^2(r-1)^2}{E}+
			\epsilon_{6,N}\frac{(s-1)^2}{E}+
			\epsilon_{7,N}\frac{(r-1)^2}{E}
	\end{equation*}
	\centerline{where}
		\begin{equation*}
			\epsilon_{i\in\{2,3,...,8\},N}=
			\frac{\epsilon_i}{\epsilon_1}
	\end{equation*}
\end{minipage}

At this point an error term could be added, and it would likely make the results from linear regression better, but doing this would change the meaning of equation \ref{eq:VolumeBaseSurfaceEquation} in such a way that it would no longer follow the effort-fatigue model. The result of adding an error term and solving back to match the form of equation \ref{eq:VolumeBaseSurfaceEquation} is shown below. This equation has relationships between $F_{tot}$ and $E_{tot}$ that are not present in the effort-fatigue model. For this reason, an error term will not be added.

\begin{equation*}
	I^2=\frac{
		\epsilon_1E
	}{
		\epsilon\epsilon_1E
		\epsilon_8+
		\epsilon_2 F_l+
		\epsilon_3 F_w+
		\epsilon_4 F_e+
		\epsilon_5 (s-1)^2(r-1)^2+
		\epsilon_6 (s-1)^2+
		\epsilon_7 (r-1)^2
	}=\frac{E_{tot}}{\epsilon E_{tot}+F_{tot}}
\end{equation*}

Throughout the rest of this chapter only the $\epsilon_{i\in\{2,3,...,8\},N}$ constants will be used. There is no need to keep the original constants as they provide no intuitive meaning beyond what can already be found in the new constants. As such, these new constants will simply replace the old constants in notation and be referred to as $\epsilon_{i\in\{2,3,...,8\}}$. The old constants will not be used. With this, the final equation for the volume base surface is shown in equation \ref{eq:VolumeBaseSurfaceEquationInverse}.

\begin{minipage}{\textwidth}
	\begin{equation}
		\frac{1}{I^2}=
			\epsilon_{8}\frac{1}{E}+
			\epsilon_{2}\frac{F_l}{E}+
			\epsilon_{3}\frac{F_w}{E}+
			\epsilon_{4}\frac{F_e}{E}+
			\epsilon_{5}\frac{(s-1)^2(r-1)^2}{E}+
			\epsilon_{6}\frac{(s-1)^2}{E}+
			\epsilon_{7}\frac{(r-1)^2}{E}
		\label{eq:VolumeBaseSurfaceEquationInverse}
	\end{equation}
	\centerline{where}
	\begin{equation*}
		\begin{split}
	        E & \in \{ 1,1.5,2,2.5, \dots ,10 \} \\
	        \epsilon, \epsilon_1, \epsilon_2, \epsilon_3, \epsilon_4, \epsilon_5,\epsilon_6,\epsilon_7 & > 0 \\
	    \end{split}
	\end{equation*}
\end{minipage}

Linear regression can now be run with this equation. The error equation that needs to be minimized is shown below.

\begin{equation*}
	E_{rr}=\sum_{
            \substack{i=0\\ t_i\in \{ T \}}
        }^n \left(
        \frac{1}{I_i^2}
        -\epsilon_8 \frac{1}{E_i}
        -\epsilon_2 \frac{F_{l,i}}{E_i}
        -\epsilon_3 \frac{F_{w,i}}{E_i}
        -\epsilon_4 \frac{F_{e,i}}{E_i}
        -\epsilon_5 \frac{(s_i-1)^2(r_i-1)^2}{E_i}
        -\epsilon_6 \frac{(s_i-1)^2}{E_i}
        -\epsilon_7 \frac{(r_i-1)^2}{E_i}
    \right)^2
\end{equation*}

To minimize the error, the partial derivatives of each unknown constant need to be found and each one set equal to zero. An example with $\epsilon_8$ is shown below.

\begin{equation*}
    \begin{split}
        \frac{\partial E_{rr}}{\partial \epsilon_8}=
        \frac{\partial}{\partial \epsilon_8}\sum_{
                \substack{i=0\\ t_i\in \{ T \}}
            }^n \left(
            \frac{1}{I_i^2}
            -\epsilon_8 \frac{1}{E_i}
            +\epsilon_2 \frac{F_{l,i}}{E_i}
            +\dots
            +\epsilon_6 \frac{(s_i-1)^2}{E_i}
            +\epsilon_7 \frac{(r_i-1)^2}{E_i}
        \right)^2&=0
        \\
        -2\sum_{
                \substack{i=0\\ t_i\in \{ T \}}
            }^n \left(
            \frac{1}{I_i^2}
            -\epsilon_8 \frac{1}{E_i}
            -\epsilon_2 \frac{F_{l,i}}{E_i}
            -\dots
            -\epsilon_6 \frac{(s_i-1)^2}{E_i}
           	-\epsilon_7 \frac{(r_i-1)^2}{E_i}
        \right)\left( \frac{1}{E_i} \right)&=0 
        \\
        \epsilon_8 \sum_{\substack{i=0\\ t_i\in \{ T \}}}^n \frac{1}{E_i^2} 
        +\epsilon_2 \sum_{\substack{i=0\\ t_i\in \{ T \}}} \frac{F_{l,i}}{E_i}
        +\dots
        +\epsilon_6 \sum_{\substack{i=0\\ t_i\in \{ T \}}}^n \frac{(s_i-1)^2}{E_i}
        +\epsilon_7 \sum_{\substack{i=0\\ t_i\in \{ T \}}}^n \frac{(r_i-1)^2}{E_i}
        &=
        \sum_{\substack{i=0\\ t_i\in \{ T \}}}^n \frac{1}{I_i^2E_i}
        \\
    \end{split}
\end{equation*}

After a similar process is carried out for each unknown constant, equation \ref{eq:VolumeBaseLinearRegMatrixEquation} can be used, and the unknown constants can be solved for.

\begin{equation}
    \label{eq:VolumeBaseLinearRegMatrixEquation}
	\left[
    \begin{matrix}
        \sum_{i=0}^n \frac{1}{E_i^2} &
        \dots &
        \sum_{\substack{i=0\\ t_i\in \{ T \}}}^n \frac{(r_i-1)^2}{E_i^2} \\

        \sum_{\substack{i=0\\ t_i\in \{ T \}}}^n  \frac{F_{l,i}}{E_i^2} &
        \dots &
        \sum_{\substack{i=0\\ t_i\in \{ T \}}}^n \frac{F_{l,i}(r_i-1)^2}{E_i^2}\\

        \vdots &
        \ddots &
        \vdots \\
        
        \sum_{\substack{i=0\\ t_i\in \{ T \}}}^n \frac{(r_i-1)^2}{E_i^2} &
        \dots &
        \sum_{\substack{i=0\\ t_i\in \{ T \}}}^n \frac{(r_i-1)^4}{E_i^2}
    \end{matrix}
    \right]
    \left[
    \begin{matrix}
        \epsilon_8 \\ \epsilon_2 \\ \vdots \\ \epsilon_7
    \end{matrix}
    \right]
    =
	\left[
    \begin{matrix}
        \sum_{\substack{i=0\\ t_i\in \{ T \}}}^n \frac{1}{I_i^2E_i} \\
        \sum_{\substack{i=0\\ t_i\in \{ T \}}}^n \frac{F_{l,i}}{I_i^2E_i} \\
        \vdots \\
        \sum_{\substack{i=0\\ t_i\in \{ T \}}}^n \frac{(r_i-1)^2}{I_i^2E_i}  
    \end{matrix}
    \right]
\end{equation}

With equations \ref{eq:VolumeBaseSurfaceEquation}- \ref{eq:VolumeBaseLinearRegMatrixEquation} the surface is fully defined and can be fitted to the data presented in chapter \ref{sec:DataSection}. Equation \ref{eq:VolumeBaseIntensitySubedInVolume} shows the result of solving the potential surface for $I$ and substituting it in equation \ref{eq:IntensityBasedVolumeEquation}. Figure \ref{fig:VolumeBaseSquatPotentialSurfaceAcrossEffort} shows the volume base potential surface and figure \ref{fig:VolumeBaseSquatPotentialSurfaceVolumeAcrossEffort} shows the volume surface generated from this surface.

\begin{equation}
    \label{eq:VolumeBaseIntensitySubedInVolume}
    \begin{split}
    		v = & srI l_{1RM} \\
    			= & l_{1RM} sr \left( 
    			\frac{
    				E
    			}{
    				\epsilon_8+
    				\epsilon_2 F_l+
    				\epsilon_3 F_w+
    				\epsilon_4 F_e+
    				\epsilon_5 (s-1)^2(r-1)^2+
    				\epsilon_6 (s-1)^2+
    				\epsilon_7 (r-1)^2
    			}
    		\right)^{\frac{1}{2}}
    \end{split}
\end{equation}

\begin{figure}[htbp]
    \centering
    \includegraphics[scale=0.55]{images/ch3/PotentialSurface/DualSquat.Effort[5,10].volumeBase.png}
    \caption{The potential surface fitted to squat data at various effort levels. The window and time frame values will be discussed in chapter \ref{sec:TimeFrame}. For now, the $I(1,1)$ and MSE values are the most important.}
    \label{fig:VolumeBaseSquatPotentialSurfaceAcrossEffort}
\end{figure}

\begin{figure}[htbp]
    \centering
    \includegraphics[scale=0.55]{images/ch3/Volume/DualSquat.Effort[5,10].volumeBase.png}
    \caption{The volume surface resulting from the potential surface fitted to squat data at various effort levels.}
    \label{fig:VolumeBaseSquatPotentialSurfaceVolumeAcrossEffort}
\end{figure}

\section{Analysis of Property 1.a: Volume and Intensity}
\label{sec:VolumeBasePotentialSurfaceProperty1a}

\section{Analysis of Property 1.b: Volume and Intensity}
\label{sec:VolumeBasePotentialSurfaceProperty1b}

\section{Analysis of Property 2: Volume and Effort}
\label{sec:VolumeBasePotentialSurfaceProperty2}

It needs to be shown that volume increases with increased effort and decreases with decreased effort. Changes in $v$ in relation to changes to $E$ is required. Solving for the partial derivative of equation \ref{eq:VolumeBaseIntensitySubedInVolume} with respect to $E$ is required.

\begin{equation*}
	\begin{split}
		\frac{\partial v}{\partial E} & =
		\frac{\partial}{\partial E} l_{1RM} sr \left( 
    			\frac{
    				E
    			}{
    				\epsilon_8+
    				\epsilon_2 F_l+
    				\epsilon_3 F_w+
    				\epsilon_4 F_e+
    				\epsilon_5 (s-1)^2(r-1)^2+
    				\epsilon_6 (s-1)^2+
    				\epsilon_7 (r-1)^2
    			}
    		\right)^{\frac{1}{2}}
    		\\
    		& = \frac{
    			l_{1RM} sr
    		}{
    			2 \left(\epsilon_8+F_{tot} \right)^{\frac{1}{2}}
    		} E^{-\frac{1}{2}}
    	\end{split}
\end{equation*}

In order for the desired behavior in effort to be observed, $\partial_E v$ needs to be $>0$. The following constraints are given in section \ref{sec:UnitsOfMeasurement}.

\begin{equation*}
	\begin{split}
		s & \ge 1 \\
		r & \in\{\mathbb{R} \ge 1 \} \\
		w & >0 \\
		E & \in\{1,1.5,2,...,10\}
	\end{split}
\end{equation*}

Given the above constraints and remembering that $l_{1RM}$ is a weight only leaves one term left, the parenthetical grouping on the bottom of the fraction. This term must be positive to satisfy the constraint on $\partial_E v$. Given this, the below statement can be made.

\begin{equation*}
	\frac{
    			l_{1RM} sr
    		}{
    			2 \left(\epsilon_8+F_{tot} \right)^{\frac{1}{2}}
    		} E^{-\frac{1}{2}} >0 \textbf{ iff }\epsilon_8+F_{tot}>0
\end{equation*}

It will be proven that the parenthetical grouping is positive in section \ref{sec:VolumeBasePotentialSurfaceProperty3}. This finalizes the proof that volume increases with increased effort and means the volume base surface properly exhibits property 2.

Before moving on, there are two things of note. The first is that this result did not depend on the sign of any constant, as there was no constant associated with the effort term. This is much different behavior from the basic surface which had to constrain the constant associated with the effort term. The reason for this can be traced all the way back to when the new constants were created to make linear regression possible. When this was done the $\epsilon_1$ constant that was associated with the effort term was absorbed into the new constants, removing the need for any constraint on $\epsilon_1$. However, doing this only moved the problem elsewhere as constraints will now need to be placed on all the constants that absorbed the $\epsilon_1$ constant. As stated previously, this work is done in section \ref{sec:VolumeBasePotentialSurfaceProperty3}.

The second thing of note is how volume increases with increases in effort. To analyze this $\partial_{EE} v$ is needed.

\begin{equation*}
	\partial_{EE}v=-\frac{
    			l_{1RM} sr
    		}{
    			4 \left(\epsilon_8+F_{tot} \right)^{\frac{1}{2}}
    		} E^{-\frac{3}{2}}
\end{equation*}

Since $\partial_{EE}v$ has all the same constants a $\partial_Ev$ it means that $\partial_{EE}v<0$. This means that volume increases with increase effort but it does so with diminishing returns. In other words, a linear increase in effort will cause smaller and smaller increases in volume at higher effort levels. This is very different from the basic surface which had linear increases in volume from linear increases in effort. The question now becomes which one is correct. Fortunately, this does not have to be answered now as the volume base model is actually flexible enough to also represent linear returns to effort. Depending on the scale of the constants from linear regression relative to the finite scale of effort, linear returns can be approximated. Not only does this add an additional layer of flexibility to the volume base model, but it will allow questions related to how volume scales with effort to be answered later in this book.

\section{Analysis of Property 3: Volume and Fatigue}
\label{sec:VolumeBasePotentialSurfaceProperty3}

It needs to be shown that volume decreases with increased fatigue and increases with decreased fatigue. To prove this it will need to be shown that the partial derivatives with respect to every fatigue term are $<0$. Shown below is $\partial_{F_l}v$.

\begin{equation*}
	\begin{split}
		\frac{\partial v}{\partial F_l} & =
		\frac{\partial}{\partial F_l} l_{1RM} sr \left( 
    			\frac{
    				E
    			}{
    				\epsilon_8+
    				\epsilon_2 F_l+
    				\epsilon_3 F_w+
    				\epsilon_4 F_e+
    				\epsilon_5 (s-1)^2(r-1)^2+
    				\epsilon_6 (s-1)^2+
    				\epsilon_7 (r-1)^2
    			}
    		\right)^{\frac{1}{2}}
    		\\
    		& = -\frac{1}{2}
    			\epsilon_2 l_{1RM} sr E^{\frac{1}{2}}
    			\left(
	    			\epsilon_8+
    				\epsilon_2 F_l+
    				\epsilon_3 F_w+
    				\epsilon_4 F_e+
    				\epsilon_5 (s-1)^2(r-1)^2+
    				\epsilon_6 (s-1)^2+
    				\epsilon_7 (r-1)^2
    			\right)^{-\frac{3}{2}}
    	\end{split}
\end{equation*}

Given the same set of constants that was presented in the previous section it should be clear that there are two terms that can change the sign of the partial derivative. The $\epsilon_2$ constant and the entire parenthetical grouping. The even root on the parenthetical grouping will technically guarantee that term is positive, but in order to make this proof more robust to future changes an alternative proof will be used. Two scenarios will be considered.

The first scenario is when $\epsilon_2>0$. If this is the case then in order for $\partial_{F_l}v<0$ the summation of the rest of the terms inside the parenthetical grouping must also be $>0$. Given the constraints on the variables from section \ref{sec:UnitsOfMeasurement} the sign of each term entirely depends on the constant in front of each term. Now, it must be remembered that this proof will eventually need to be applied to every term in $F_{tot}$. This means that every term will depend on the sign of every other term in $F_{tot}$. This removes the scenario where some terms could be negative and the rest could be positive as $\partial_{F_l}v$ would not be $<0$ for the negative terms because the parenthetical grouping would still be $>0$ despite the sign of the constant changing. This forces $\epsilon_2$-$\epsilon_7$ to all be $>0$ in order for $\partial_{F_l}v<0$. 

It is also important to consider $\epsilon_8$, as it is not part of $F_{tot}$ but is still part of the parenthetical grouping that could affect the sign of $\partial_{F_l}v$. Initial intuition would limit $\epsilon_8>-F_{tot}$, but this creates a dependence between the constant $\epsilon_8$ and the independent variables. Instead, $\epsilon_8$ just needs to be constrained to being greater than the smallest value of $F_{tot}$, which given the range of the variables as well as the constraints placed on $\epsilon_2$-$\epsilon_7$ the minimum value is zero. It is important to stop in the middle of all this theory and note that this minimum value matches what would be expected in reality.

The second scenario is if $\epsilon_2<0$. If this is the case then in order for $\partial_{F_l}v<0$ the summation of the rest of the terms inside the parenthetical grouping must also be $<0$. \footnote{This also assumes that the root is not even so negative values can be produced.} Similar to the previous scenario, if $\epsilon_2<0$ then it forces $\epsilon_2$-$\epsilon_7$ to all be $<0$. While this case does still allow for the proper behavior between volume and fatigue, it allows volume to be negative. \footnote{Again, assuming a non-even root that allows negative volume.} This negative volume could easily be counteracted by forcing $\epsilon_8>-F_{tot}$ but this creates a dependence between the constant $\epsilon_8$ and the independent variables, making this approach invalid. Unlike the previous scenario, there is not minimum value for $F_{tot}$ in this scenario due to $\epsilon_2$-$\epsilon_7$ all being $<0$. Due to $F_{tot}$ not having a minimum there is no way to constrain volume to being $>0$, making this scenario invalid.

After all of that discussion, it can be concluded that after constraining $\epsilon_2$-$\epsilon_8$ to be $>0$ the volume base surface properly exhibits property 3. Just like the previous section further analysis can be done that looks at how volume decreases from increases in fatigue. Again, this needs to be done for every term in $F_{tot}$. Shown below is $\partial_{F_lF_l}v$.

\begin{equation*}
	\partial_{F_lF_l}v=
		\frac{3}{4}
    		\epsilon_2^2 l_{1RM} sr E^{\frac{1}{2}}
    		\left(
	    		\epsilon_8+
    			\epsilon_2 F_l+
    			\epsilon_3 F_w+
    			\epsilon_4 F_e+
    			\epsilon_5 (s-1)^2(r-1)^2+
    			\epsilon_6 (s-1)^2+
    			\epsilon_7 (r-1)^2
    		\right)^{-\frac{5}{2}}
\end{equation*}

Just like the previous section, the second partial derivative changed sign relative to the first partial derivative. In this case it means that volume decreases with diminishing returns to increases in fatigue. \footnote{Having a linear increase in fatigue represent a sub-linear decrease in volume can have some very important implications when it comes to planning deloads for lifters. It could be said that this is why lifters make progress, as a lifter would not have to linearly decrease volume every time fatigue is high to where they were previously, allowing for incremental increases in total volume over time.} Also just like the last section, linear returns can be approximated.


\section{Analysis of Property 4: Sets and Reps}
\label{sec:VolumeBasePotentialSurfaceProperty4}

\section{Analysis of Property 5: Bounded Volume}
\label{sec:VolumeBasePotentialSurfaceProperty5}

\section{Analysis of Property 6: 1RM Estimations}
\label{sec:VolumeBasePotentialSurfaceProperty6}

\section{Summary}
