\chapter{
    Data
    \\
    \large{The Source of Truth}
}
\label{sec:DataSection}

The data associated with this paper is present in the projects associated GitHub repository, available at the following url:  \url{github.com/barbell-math/data}.
The data represents almost two years worth of training data, all collected from a single male lifter. There are over 900 data points in this data set varying across many different exercises. Before recording any data, the lifter was asked to list there 1RM's for several lifts. These were recorded as having been performed on macrocycle 1 and having RPE 10.

The data recorded consists of a list of exercises performed at the gym, with the lifter recording the following data for each exercise:

\begin{itemize}
    \item What exercise was performed
    \item The weight the exercise was performed at
    \item The number of sets that were performed
    \item The number of reps that were performed
    \item The date the exercise was performed on
    \item The RPE the exercise required
\end{itemize}

In addition to the data listed above, the lifter also recorded the starting and ending dates of each macrocycle. From the data the lifter recorded, the following additional data points were calculated. These data points were calculated for the lifter in order to avoid any errors from manual entry.

\begin{itemize}
    \item The volume for each exercise was obtained by multiplying the sets, reps, and weight. This comes directly from equation \ref{eq:BaseVolumeEquation}.
    \item The intensity for each exercise was calculated in relation to the best lift of the same exercise in the previous macrocycles. This comes directly from equation \ref{eq:BaseIntensityEquation}.
    \item Each lift was given a macrocycle ID according to the date the exercise was performed.
\end{itemize}

During the time period the lifter was recording data, there were several notable occurrences:

\begin{enumerate}
    \item Near 3/2/2022 the lifter sustained an injury to his lower back, specifically his sacroiliac joint, or SI joint. This injury was a result of rounding of the lumbar spine while attempting a maximal effort deadlift, and required chiropractic care coupled with 2 weeks off of training. The failed lift was not recorded.
    
    \item On 5/5/2022 the lifter participated in a deadlift only competition. The 5 week prep leading up to that competition was very successful, and resulted in the lifter getting a 20 lb PR on the deadlift.
    
    \item On 7/24/2022 the lifter participated in a full meet. The prep leading up to that competition was very consistent, and resulted in the lifters best performance on the platform to date.
    
   	\item On 9/10/2022 the lifter participated in a full meet. The lifter set 3/5 Colorado state records in his weight and age class, and matched the other two.
    
    \item On 10/5/2022 the lifter tore his hamstring squatting. The diagnosis was a grade 1-2 muscle belly tear, and the lifter went to three physical therapy (PT) sessions before feeling comfortable to adjust training on his own.
\end{enumerate}

\section{Units of Measurement}
\label{sec:UnitsOfMeasurement}

Before starting, it is worth mentioning how elements in the data set were measured as it will set the stage for future discussion.

Weight, and hence intensity, are obvious. Recording the weight lifted and the fraction that weight is of the lifters 1RM is all that's required. This defines weight as $w> 0$ and intensity as $I>0$. Weight is not allowed to equal $0$ because that would imply there is no resistance and hence no training stimulus, making the exercise useless. \footnote{Sometimes people classify body weight exercises as having $0$ weight, despite still having to lift some fraction of there own body weight. This is mainly done because it can be difficult to measure the exact proportion of there body weight they are actually lifting.} Intensity is not capped at $1$ because lifting a weight greater than the lifters previous 1RM is the ultimate goal of a powerlifting program.

Reps are just positive integer values greater than or equal to $1$, or $r\in \{ \mathbb{R}\ge 1 \}$. 
As discussed in section \ref{sec:FractionalSets}, sets are in the domain of positive numbers greater than or equal to $1$, or $s\ge 1$. Effort will follow the RPE scale, discussed in \ref{sec:CommonTermsSection}, defining it as $E\in \{0,0.5,1,...,10\}$.

Time will be recorded as a date, which means it will have units of days. Dates by themselves cannot be mathematically used. To work around this, the past will be negative values representing the number of days since the exercise was performed, making $0$ represent the current day. Future dates can (and will) be considered, and will be represented as positive values. Given this, time will be in the domain of integer values, or $t\in \{ \mathbb{R} \}$.

Following the traditional definition, frequency will be calculated across days, meaning separate sets of the same exercise on the same day will not count toward an increased frequency. Separate sets of the same exercise on different days will increase the frequency of the exercise. Naturally, frequency is limited to positive integer values, or $f\in \{ \mathbb{R}\ge 0 \}$.

\section{Small Differences: What's Different From 'Standard'}
\label{sec:SmallDifferencesSection}

The vast majority of concepts surrounding lifting that are used in this book will not differ from there standard definitions introduced in section \ref{sec:CommonTermsSection}. However, due to the nuanced detail that modeling requires, some small tweaks were made when recording data. Each of these small differences will be discussed along with why changes were made.

\subsection{Set Wise RPE Grouping}
\label{sec:SetWiseRPEGrouping}

One small change will be made to the 'standard' RPE system. Typically, RPE is measured for each set a lifter completes, regardless of any other parameters that dictate what is done for an exercise. In the context of this book, RPE will be recorded only once for each unique combination of sets, reps, and weight performed during an exercise. The recorded RPE will match the highest RPE required to perform all of the sets. Typically this will just be the RPE of the last set, but is not guaranteed to be the last set. An example may help demonstrate. The first part of a hypothetical deadlift workout with the 'standard' RPE scale is shown in table \ref{tab:StandardRPEExample}.

\begin{table}[h]
	\centering
	\begin{tabular}{c|c|c|c|c|c}
		Date & Exercise & Sets & Reps & Weight & Effort \\
        \hline
        Mon, July 4\textsuperscript{th} & Deadlifts & $1$ & $1$ & $455$ lbs & $8.5$ \\
        Mon, July 4\textsuperscript{th} & Deadlifts & $1$ & $4$ & $405$ lbs & $6$ \\
        Mon, July 4\textsuperscript{th} & Deadlifts & $1$ & $4$ & $405$ lbs & $6$ \\
        Mon, July 4\textsuperscript{th} & Deadlifts & $1$ & $4$ & $405$ lbs & $6.5$ \\
        Mon, July 4\textsuperscript{th} & Deadlifts & $1$ & $4$ & $405$ lbs & $7.5$ \\
        Mon, July 4\textsuperscript{th} & Deadlifts & $1$ & $4$ & $405$ lbs & $8.5$ \\
        Mon, July 4\textsuperscript{th} & Deadlifts & $1$ & $10$ & $315$ lbs & $7$ \\
	\end{tabular}
	\caption{A table demonstrating the 'standard' way RPE is measured, on a per-set level. Note how the RPE varies across sets with the same set, rep, and weight values, likely because the lifter got tired as the workout continued.}
	\label{tab:StandardRPEExample}
\end{table}

Now, table \ref{tab:NonStandardRPEExample} demonstrates the adjusted RPE system that this book uses.

\begin{table}[h]
	\centering
	\begin{tabular}{c|c|c|c|c|c}
		Date & Exercise & Sets & Reps & Weight & Effort \\
        \hline
        Mon, July 4\textsuperscript{th} & Deadlifts & $1$ & $1$ & $455$ lbs & $8.5$ \\
        Mon, July 4\textsuperscript{th} & Deadlifts & $5$ & $4$ & $405$ lbs & $8.5$ \\
        Mon, July 4\textsuperscript{th} & Deadlifts & $1$ & $10$ & $315$ lbs & $7$ \\
	\end{tabular}
	\caption{A table demonstrating the way RPE is measured for this book. Note how only the highest RPE value is recorded for distinct set, rep, and weight combinations.}
	\label{tab:NonStandardRPEExample}
\end{table}

This may seem like an odd choice but there are several reasons for doing this. The most obvious, and least important reason, is it matches what the lifter would see in there training program. A lifter is typically told to do something like '$5$ sets of $4$ at $85$\%'. Given this terminology, it would make more sense to the lifter to only record one RPE value and not $5$. More importantly however, is grouping common sets, reps, and weights gives the model a more accurate view to learn from. Given the 'standard' way of recording RPE, there is no need to record sets at all, as there will only ever be one set, each one with a varying RPE value. Without getting too deep into the math, this will also create problems later when attempting to learn from the data, as the same rep and weight values will have different RPE values, removing correlations in the data and rendering predictions useless. By only recording the highest RPE and allowing the number of sets to increase past $1$, the model gets a far clearer representation of what the lifter is actually doing.

It is worth mentioning why the max RPE of a particular set, rep, and weight combination is chosen. Other measures such as the average RPE could also be used. The simple answer is that it does not make sense to average effort. Averaging effort could lead to scenarios where the max RPE is $10$ but the average is less than $10$, leading the model to assume that more volume could be done despite some sets already requiring maximum effort. RPE $10$ cannot be surpassed but it would be required to be passed if more volume were prescribed. Selecting the max RPE instead of the average avoids this problem.

\subsection{Fractional Sets}
\label{sec:FractionalSets}

Sets appear to only require integer values, but a specific case leads to a different representation. Lets say a lifter is prescribed to squat for $5$ sets of $3$ with the same weight across all $5$ sets. The exact weight is not relevant to the example. Then lets say the lifter completed all $3$ reps on the first $4$ sets, but only managed to complete $2$ reps on the last set. One way to record this is shown in table \ref{tab:FailedSetExampleIncorrectData}.

\begin{table}[h]
    \centering
    \begin{tabular}{c|c|c|c}
        Exercise & Sets & Reps & \dots \\
        \hline
        Squat & $4$ & $3$ & \dots \\
        Squat & $1$ & $2$ & \dots \\ 
    \end{tabular}
    \caption{A table illustrating the incorrect way to record failed reps across sets.}
    \label{tab:FailedSetExampleIncorrectData}
\end{table}

Recording the missed rep this way will lead to problems later when attempting to fit a surface to the data, as a single data point will have turned into two correlated data points. \footnote{Linear regression assumes that the data points are independent from each other.} Put another way, the same exercise now has two data points. To remedy this, fractional sets will be used, making sets in the domain of positive numbers greater than or equal to $1$, or $s\ge 1$. The reason sets cannot be less than $1$ is because any fractional sets less than one can simply be represented as a single set of less reps. Table \ref{tab:FailedSetExampleCorrectData} shows the adjusted way to record failed reps.

\begin{table}[h]
    \centering
    \begin{tabular}{c|c|c|c}
        Exercise & Sets & Reps & \dots \\
        \hline
        Squat & $4\frac{2}{3}$ & 3 & \dots \\
    \end{tabular}
    \caption{A table illustrating the correct way to record failed reps across sets.}
    \label{tab:FailedSetExampleCorrectData}
\end{table}

Note that volume, as shown below, is not changed by making sets fractional. Intensity is not changed in virtue of the weight not changing and frequency is also not changed due to the sets all being performed on the same day.

\begin{equation*}
    \begin{split}
        v_1=&v(4,3,w)+v(1,2,w)=14w \\
        v_2=&v\left(4\frac{2}{3},3,w\right)=14w \\
    \end{split}
\end{equation*}