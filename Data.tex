\chapter{
    Data
    \\
    \large{The Source of Truth}
}
\label{sec:DataSection}

The data associated with this paper is present in the projects associated GitHub repository, available at the following url:  \url{github.com/carmichaeljr/powerlifting-engine}.
The data represents almost two years worth of training data, all collected from a single male lifter. There are over 900 data points in this data set varying across many different exercises. Before recording any data, the lifter was asked to list there 1RM's for several lifts. These were recorded as having been performed on macrocycle 1 and having RPE 10.

The data recorded consists of a list of exercises performed at the gym, with the lifter recording the following data for each exercise:

\begin{itemize}
    \item What exercise was performed
    \item The weight the exercise was performed at
    \item The number of sets that were performed
    \item The number of reps that were performed
    \item The date the exercise was performed on
    \item The RPE the exercise required
\end{itemize}

In addition to the data listed above, the lifter also recorded the starting and ending dates of each macrocycle. From the data the lifter recorded, the following additional data points were calculated. These data points were calculated for the lifter in order to avoid any errors from manual entry.

\begin{itemize}
    \item The volume for each exercise was obtained by multiplying the sets, reps, and weight. This comes directly from equation \ref{eq:BaseVolumeEquation}.
    \item The intensity for each exercise was calculated in relation to the best lift of the same exercise in the previous macrocycles. This comes directly from equation \ref{eq:BaseIntensityEquation}.
    \item Each lift was given a macrocycle ID according to the date the exercise was performed.
\end{itemize}

During the time period the lifter was recording data, there were several notable occurrences:

\begin{enumerate}
    \item Near 3/2/2022 the lifter sustained an injury to his lower back, specifically his sacroiliac joint, or SI joint. This injury was a result of rounding of the lumbar spine while attempting a maximal effort deadlift, and required chiropractic care coupled with 2 weeks off of training. The failed lift was not recorded.
    
    \item On 5/5/2022 the lifter participated in a deadlift only competition. The 5 week prep leading up to that competition was very successful, and resulted in the lifter getting a 20 lb PR on the deadlift.
    
    \item On 7/24/2022 the lifter participated in a full meet. The prep leading up to that competition was very consistent, and resulted in the lifters best performance on the platform to date.
    
   	\item On 9/10/2022 the lifter participated in a full meet. The lifter set 3/5 Colorado state records in his weight and age class, and matched the other two.
    
    \item On 10/5/2022 the lifter tore his hamstring squatting. The diagnosis was a grade 1-2 muscle belly tear, and the lifter went to three physical therapy (PT) sessions before feeling comfortable to adjust training on his own.
\end{enumerate}

\section{Units of Measurement}
\label{sec:UnitsOfMeasurement}

Before starting, it is worth mentioning how elements in the data set were measured as it will set the stage for future discussion.

Weight, and hence intensity, are obvious. Recording the weight lifted and the fraction that weight is of the lifters 1RM is all that's required. This defines weight as $w> 0$ and intensity as $I>0$. Weight is not allowed to equal $0$ because that would imply there is no resistance and hence no training stimulus, making the exercise useless. \footnote{Sometimes people classify body weight exercises as having $0$ weight, despite still having to lift some fraction of there own body weight. This is mainly done because it can be difficult to measure the exact proportion of there body weight they are actually lifting.} Intensity is not capped at $1$ because lifting a weight greater than the lifters previous 1RM is the ultimate goal of a powerlifting program.

Reps are just positive integer values greater than or equal to $1$, or $r\in \{ \mathbb{R}\ge 1 \}$. 
As discussed in section \ref{sec:FractionalSets}, sets are in the domain of positive numbers greater than or equal to $1$, or $s\ge 1$. Effort will follow the RPE scale, discussed in \ref{sec:CommonTermsSection}, defining it as $E\in \{0,0.5,1,...,10\}$.

Time will be recorded as a date, which means it will have units of days. Dates by themselves cannot be mathematically used. To work around this, the past will be negative values representing the number of days since the exercise was performed, making $0$ represent the current day. Future dates can (and will) be considered, which will be represented as positive values. Given this, time will be in the domain of integer values, or $t\in \{ \mathbb{R} \}$.

Following the traditional definition, frequency will be calculated across days, meaning separate sets of the same exercise on the same day will not count toward an increased frequency. Separate sets of the same exercise on different days will increase the frequency of the exercise. Naturally, frequency is limited to positive integer values, or $f\in \{ \mathbb{R}\ge 0 \}$.